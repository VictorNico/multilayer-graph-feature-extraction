\documentclass{ltxdockit}
\usepackage{a4wide}
\usepackage{hyperref}
\usepackage[T1]{fontenc}
\usepackage[utf8]{inputenc}
\usepackage{alltt}
\usepackage{listings}
\usepackage{shortvrb}
\MakeShortVerb{\|}

\titlepage{%
  title={Bib\LaTeX\ style extension for Software},
  subtitle={Citation and References macros for Bib\LaTeX},
  url={http://mirror.ctan.org/macros/latex/exptl/biblatex-contrib/biblatex-software/},
  author={Roberto Di Cosmo},
  email={roberto@dicosmo.org},
  revision={1.0},
  date={\today}}

\hypersetup{%
  pdftitle={Bib\LaTeX\ style extension for Software},
  pdfsubject={Style for Bib\LaTeX},
  pdfauthor={Roberto Di Cosmo},
  pdfkeywords={latex, biblatex, software, style}}

\begin{document}

\printtitlepage
\tableofcontents

\section{Introduction}\label{sec:intro}

\subsection{About}

Software plays a significant role in modern research, and it must be properly
acknowledged and referenced in scholarly works. To this end, specific
bibliographic entries for describing \emph{software}, \emph{software versions},
\emph{software modules} and \emph{code fragments} have been designed by a
dedicated task force at Inria in 2020 that brought together researchers from
Computer Science and Applied Mathematics to discuss best practices for archiving
and referencing software source code~\cite{2020GtCitation}.\\

This package is a Bib\LaTeX\ \emph{style extension} that adds support for these
four \emph{software entry types} to any other Bib\LaTeX\ style used in documents
typeset in \latex. It is made up of the following key components: a references
section style (\path{software.bbx}), a data model extension
(\path{software.dbx}) and string localisation files
(\path{<language>-software.lbx})\footnote{String localisations are provided for
  some languages: localisations for other languages are welcome, feel free to
  contribute them on the official repository, see Section \emph{Contributing} below.}.\\

The distribution contains other material, for demonstration purposes, and for
more advanced use.

\subsection{License}

Permission is granted to copy, distribute and/or modify this software under
the terms of the \latex Project Public License, version
1.3c\footnote{\url{http://www.latex-project.org/lppl.txt}}. The current
maintainer is Roberto Di Cosmo (\textcopyright\ 2020).

\subsection{History}

When I decided to start the Software Heritage initiative in 2015, software in
general and research software in particular was not yet a first class citizen in
the scholarly world. The absence of support for properly citing software in a
bibliography was just one of the many signs of this lack of recognition.

In order to properly reference a software project, and ensure that these
references are stable enough to pass the test of time, it was necessary to build
a \emph{universal archive} for software source code, and to equip every software
artifact with \emph{intrinsic} identifiers.

Now that \href{https://www.softwareheritage.org}{Software Heritage} is providing
the universal archive~\cite{swhcacm2018}, with Software Heritage \emph{intrinsic
  identifiers} (SWH-IDs) widely available~\cite{cise-2020-doi}, it is finally
possible to propose proper bibliographic entries for software, at various levels
of granularity, down to the line of code.

\subsection{Acknowledgments}

Thanks to the Inria working group members for their precious feedback and
contribution to the desing of the software bibliography entries: Pierre Alliez,
Benjamin Guedj, Alain Girault, Morane Gruenpeter, Mohand-Said Hacid, Arnaud
Legrand, Xavier Leroy, Nicolas Rougier and Manuel Serrano.

\section{Software entries}
\input{swentries}

\section{Use}\label{ref:use}
\label{use}

This package can be used as a standalone on the fly extension, or to produce
full bibliographic styles that extend pre-existing styles.

\subsection{Use as an \emph{on the fly} extension}

The simplest way to use this package is to follow the example given in the
\path{sample-use-sty.tex} that shows how one can \emph{extend on the fly} any
existing Bib\LaTeX\ style by just doing the following:

\begin{itemize}
\item pass the \texttt{datamodel=software} option to the \texttt{biblatex} package as in
 \begin{ltxcode}
   \usepackage[datamodel=software]{biblatex}
 \end{ltxcode}
 \item load the software biblatex style using
 \begin{ltxcode}
   \usepackage{software-biblatex}
 \end{ltxcode}
 \item set software specific bibliography options using the macro |\ExecuteBibliographyOptions|;
   the options with their default values are as in
 \begin{ltxcode}
   \ExecuteBibliographyOptions{
     halid=true,
     swhid=true,
     swlabels=true,
     vcs=true,
     license=true}
 \end{ltxcode}
\end{itemize}

This is quite handy to add support for software entries in a single article, as
it is enough to add \path{software.dbx}, \path{software.bbx},
\path{<language>-software.lbx} and \path{software-biblatex.sty} to make it work.

\subsection{Generate extended styles}

When a more systematic use is foreseen, as for institution-wide reports, or
conference and journal proceedings, it is more appropriate to generate a new
biblatex style that includes support for the software entries right away.

The following simple mechanism is provided for this use case:

\begin{itemize}
\item add to the \path{stublist} file the names of all the existing styles that must be extended
\item run \texttt{make biblatex-styles} to produce new style files, with an added \texttt{+sw} suffix,
   for each of the existing style
\item install the newly generated files in the standard path where Bib\LaTeX\ files are found
\end{itemize}

The stock \path{stublist} file contains the names of all the standard Bib\LaTeX\ 
styles.  If this approach is followed, then one can load directly the extended
file, and the software specific bibliography options become available when
loading the Bib\LaTeX\ package directly.  See the \path{sample.tex} file for a
working example.

\subsection{Installation}\label{sec:install}

This package may become available in standard distributions like \TeX Live as
|biblatex-software|. To install manually, you can download it from CTAN and
then, put the relevant files in your texmf tree, usually:\\

\noindent\path{<texmf>/tex/latex/biblatex-software/software-biblatex.sty}\\
\path{<texmf>/tex/latex/biblatex-software/software.bbx}\\
\path{<texmf>/tex/latex/biblatex-software/software.dbx}\\
\path{<texmf>/tex/latex/biblatex-software/<language>-software.lbx}\\


\subsection{Package options}\label{sec:options}

The following options are available to control the typesetting of
software related entries.

\begin{ltxcode}
  swlabels=true|false
\end{ltxcode}

\noindent Software is a special research output, distinct from
publications, hence software entries in a bibliography are
distinguished by a special label by default.
This behaviour can be disabled by setting this option to |false|.

\begin{ltxcode}
  license=true|false
\end{ltxcode}

\noindent This option controls whether license information for
the software entry is typeset. The default is |true|.

\begin{ltxcode}
  halid=true|false
\end{ltxcode}

\noindent This option controls the inclusion of the identifier on the HAL repository of the
metadata record for the software described in the entry. The default is |true|.

\begin{ltxcode}
  swhid=true|false
\end{ltxcode}

\noindent This option controls the inclusion of the identifier on the Software Heritage archive
of the source code of the software described in the entry. The default is |true|.

\begin{ltxcode}
  vcs=true|false
\end{ltxcode}

\noindent This option controls the inclusion of the url of the code hosting
platform where the software described in the enttry is developed. The default is |true|.

\section{Details}

The detailed information for this style is contained in the example document and
accompanying \path{.bib} files:
\begin{description}
\item[\path{software-biblatex.tex}] This document.
\item[\path{biblio.bib}] An example bibliography showcasing all the software entries.
\item[\path{sample-use-sty.tex}]\footnote{\path{sample-use-sty.pdf} is also
  provided and is the typeset version of this \latex source file.} This document
  exerces most useful feature of this style extension, using the
  \path{biblio.bib} entries.
\item[\path{sample.tex}] This document produces the same output as
  \path{sample-use-sty.tex}, but instead of extending on the fly and existing
  style, it assumes that an extended bibliographic style \texttt{numeric+sw} has
  been created starting from the standard \texttt{numeric} style.
\item[\path{software.bbx}] The |biblatex-software| references style.
\item[\path{software.dbx}] The |biblatex-software| data model additions.
\item[\path{*.lbx}] The |biblatex-software| localisation files.
\item[\path{software-biblatex.sty}] The |software-biblatex| \LaTeX\ package for extending on the fly any preloaded Bib \LaTeX\ style.
\end{description}

\section{Contributing}

This style extension is currently developed on a git-based repository at
\url{https://gitlab.inria.fr/gt-sw-citation/bibtex-sw-entry/}.
Contributions and bug reports are very welcome. In particular, translation
of the localization strings for other languages are needed.

\section{Revision history}\label{rev}

\begin{changelog}

\input{history}

\end{changelog}

\bibliography{manual}
\bibliographystyle{abbrv}

\end{document}
