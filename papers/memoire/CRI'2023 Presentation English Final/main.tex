\documentclass[10pt]{beamer}
\usepackage[utf8]{inputenc}

\usepackage{bm}
\usepackage{multirow,rotating}
\usepackage{color}
\usepackage{hyperref}
\usepackage{tikz-cd}
\usepackage{array}
\usepackage{siunitx}
\usepackage{mathtools,nccmath}%
\usepackage{etoolbox, xparse} 
\usetheme{CambridgeUS}%CambridgeUS Warsaw
\usecolortheme{dolphin}

% set colors
%\definecolor{myNewColorA}{RGB}{158, 27,50}
%\definecolor{myNewColorB}{RGB}{158, 27,50}
%\definecolor{myNewColorC}{RGB}{158, 27,50} % {130,138,143}
%\setbeamercolor*{palette primary}{bg=myNewColorC}
%\setbeamercolor*{palette secondary}{bg=myNewColorB, fg = white}
%\setbeamercolor*{palette tertiary}{bg=myNewColorA, fg = white}
%\setbeamercolor*{titlelike}{fg=myNewColorA}
%\setbeamercolor*{title}{bg=myNewColorA, fg = white}
%\setbeamercolor*{item}{fg=myNewColorA}
%\setbeamercolor*{caption name}{fg=myNewColorA}
%\usefonttheme{professionalfonts}

\usepackage{natbib}
\usepackage{hyperref}
%------------------------------------------------------------
% \titlegraphic{\includegraphics[height=0.75cm]{ua_eng_logo.png}} 

% logo of my university


\titlegraphic{%
%\includegraphics[width=3.0cm]{ua_seal.png}
}

\setbeamerfont{title}{size=\large}
\setbeamerfont{subtitle}{size=\small}
\setbeamerfont{author}{size=\small}
\setbeamerfont{date}{size=\footnotesize}
\setbeamerfont{institute}{size=\footnotesize}
\title[]{Time aware implicit social influence estimation to enhance recommender systems performances}%title


% \begin{figure}[ht]
%             \centering
%             \begin{minipage}{0.43\textwidth}
%                 \centering
%                 \includegraphics[width=0.25\linewidth]{logo_Univ.jpg}
%                 %\caption{Exponential decay function - EDF  }
%                 \label{fig:edf}
%             \end{minipage}\hfill
%             \begin{minipage}{0.4\textwidth}
%                 \centering
%                 \includegraphics[width=0.45\linewidth]{logo_UMMISCO.png}
%                 %\caption{Logistic decay function - LDF }
%                 \label{fig:ldf}
%             \end{minipage}\hfill
    
%         \end{figure}

%     \vspace{-1.2cm}


%\subtitle{ }%%subtitle
\author[]{ NZEKON Armel$^{1,3}$, HAMZA Adamou$^{1,3}$, MESSI Thomas$^{1,2,3}$ and BETNDAM Bleriot$^{1}$  \\
\vspace{0.3cm}
  \\
\vspace{0.3cm}
$^{1}$University of Yaounde I, FS, Computer Science Department, Cameroon \\
\vspace{0.3cm}
$^{2}$University of Ebolowa, HITLC, Computer Engineering Department, Cameroon\\
\vspace{0.3cm}
$^{3}$Sorbonne Université, IRD, UMI 209 UMMISCO, F-93143, Bondy, France\\
\vspace{0.4cm}
\includegraphics[width=0.35\textwidth]{logos_CRI_rr.png}
\vspace{-0.7cm}
%\vspace{0.3cm}
}%%authors  email: pagnaul.betndam@facsciences-uy1.cm

%\begin{figure}[ht]
%\vspace{-0.7cm}
%\includegraphics[width=0.3\textwidth]{logos_CRI_r.png}
%\vspace{-0.8cm}
%\caption{KNN results with Ciao and Epinions} \label{fig:knn_sc}
%\end{figure}

% \begin{figure}
%             \centering
%             \begin{minipage}{0.43\textwidth}
%                 \centering
%                 \includegraphics[width=0.25\linewidth]{logo_Univ.jpg}
%                 %\caption{Exponential decay function - EDF  }
%                 \label{fig:edf}
%             \end{minipage}\hfill
%             \begin{minipage}{0.4\textwidth}
%                 \centering
%                 \includegraphics[width=0.45\linewidth]{logo_UMMISCO.png}
%                 %\caption{Logistic decay function - LDF }
%                 \label{fig:ldf}
%             \end{minipage}\hfill
    
%         \end{figure}

%     \vspace{-1.2cm}

%\includegraphics[width=0.15\textwidth]{logo_Univ.jpg}

\institute[]{}
\date[\textcolor{white}{Rapport de lecture}]
{13 December 2023}
\vspace{-0.1cm}

%------------------------------------------------------------
%This block of commands puts the table of contents at the 
%beginning of each section and highlights the current section:
%\AtBeginSection[]
%{
%  \begin{frame}
%    \frametitle{Contents}
%    \tableofcontents[currentsection]
%  \end{frame}
%}
% \AtBeginSection[]{
%   \begin{frame}
%   \vfill
%   \centering
%   \begin{beamercolorbox}[sep=8pt,center,shadow=true,rounded=true]{title}
%     \usebeamerfont{title}\insertsectionhead\par%
%   \end{beamercolorbox}
%   \vfill
%   \end{frame}
% }
% ------Contents below------
%------------------------------------------------------------

\begin{document}

%The next statement creates the title page.
\frame{\titlepage}
\begin{frame}
\frametitle{PLAN}
\tableofcontents
\end{frame}


% consider removing it if it's too redundant
% \AtBeginSection[]
% {
%   \begin{frame}
%     \frametitle{Table of Contents}
%     \tableofcontents[currentsection]
%   \end{frame}
% }

%------------------------------------------------------------
\section{Introduction}
%\subsection{Context}
\begin{frame}{Context}
    \begin{center}
        \includegraphics[width=0.55\textwidth]{Context_CRI_r.png}
    \end{center}
    
\end{frame}

%\subsection{Overview}
\section{Recommender systems}
\begin{frame}{Definitions of recommender systems}
\begin{block}{Definitions}
    \begin{enumerate}
        \item According to Gediminas Adomavicius et al[6] recommender systems help to cope with information overload and provide recommendations for personalised content and services.
\vspace{1cm}
        %\item Information filtering techniques that predict the rank or preference that a user assigns to an item from a set of items that are likely to be of interest.
        \item Suggest to each user items he is most likely to like in the near future.

        %de même nature (films, musiques, livres, images, etc.)
    \end{enumerate}
    
\end{block}

\end{frame}

\subsection{Recommender system approaches}
\begin{frame}{Recommender system approaches}
\vspace{-0.4cm}
\begin{block}{Recommender system approaches}
    \begin{enumerate}
        \item \textbf{Collaborative filtering} \\
        Recommend the same items to users who have had the same preferences in the past.
        
        \item \textbf{Content-based filtering} \\
       
        Recommend items with the same characteristics as those that the target user has liked in the past.

        \item \textbf{Hybrid filtering} \\
        Taking advantage of the strengths of the two previous approaches while avoiding some of their weaknesses.
    \end{enumerate}
    
\end{block}

\vspace{-0.3cm}
\begin{block}{Limits of these approaches}
    \begin{enumerate}
    \vspace{-0.1cm}
        \item \textbf{Cold start :} new user on the platform.
        \item \textbf{Lack of data :} users interact with very few items. 
        \item \textbf{Under-use of the history of ratings that users assign to items:} timing of item purchases by users is not exploited.
        \item \textbf{The non-use of certains informations :} relationships of friendship and trust between users.
    \end{enumerate}
    
\end{block}

\end{frame}

\subsection{Trust-based recommender systems}
\begin{frame}{Trust-based recommender systems}
\begin{block}{Definitions of trust}
In the social context, \textbf{Trust} generally refers to the fact that one person trusts the words and actions of another.\\

 In the field of recommender systems, \textbf{Trust} is defined in
terms of a user’s ability to provide relevant recommendations to another user, Guo et al. [3]
\end{block}

\begin{block}{Properties of trust}
\begin{enumerate}
        \item Transitivity: if $u \rightarrow v$ and $v \rightarrow w$ then $u \rightarrow w$

        \item Asymmetry: $trust(u,v) \ne trust(v,u)$

        \item Context dependency: $trust(u,v,c_1) \ne trust(u,v,c_2)$
        \item Personalized: $trust(u,v) \ne trust(u,w)$ and $trust(u,v) \ne trust(w,v)$ 
    \end{enumerate}
    
\end{block}

Trust-based recommender systems use information about trust and friendship between users to recommend items.

\end{frame}

\subsubsection{Explicit trust-based recommender systems}
\begin{frame}{Explicit trust-based recommender systems}
\begin{enumerate}
    \item \textbf{Jian-Ping Mei , Han Yu , Zhiqi Shen and Chunyan Miao - 2017 [2]:}\\
    \begin{itemize}
        \item Explicit trust relationships between users available in the epinions dataset and the history of user ratings are exploited to calculate the trust that one user $u$ places in another $v$.  
        \item These calculated trusts are then integrated into two classical recommender systems, one based on k nearest neighbours and the other based on matrix factorisation.
    \end{itemize}
\vspace{0.3cm}
\begin{center}
    $Trust(u,v) = NTrust_v$ or $Trust(u,v) = NRating_v$\\ \\
\end{center}
        
        \vspace{0.3cm} 
     with \bm{$NTrust_v$} being the number of users who have explicitly declared that they trust $v$ and \bm{$NRating_v$} the number of purchases user $v$ has made on the e-commerce site. \\
     \vspace{0.3cm}
    \textbf{Limits:} Trust here is not personalised and does not take into account the temporal dynamics of social influence.

    % \begin{equation}
    %     Trust(u,v) = NTrust_v
    % \end{equation}
    %  with $NTrust_v$ being the number of users who have explicitly declared that they trust v.
     
    % \begin{equation}
    %     Trust(u,v) = NRating_v
    % \end{equation}
    % with $NRating_v$ being the number of purchases user $v$ has made on the e-commerce site.

    % \textbf{Limits:} trust here is not personalised and does not take into account the temporal dynamics of social influence.
    
\end{enumerate}
\end{frame}


\subsubsection{Implicit trust-based recommender systems}
\begin{frame}{Implicit trust-based recommender systems}
\begin{enumerate}[2]
    
    \item \textbf{Armel Jacques Nzekon Nzeko’o, Maurice Tchuente and Matthieu Latapy - 2019 [1]: }\\
    \begin{itemize}
        \item Jaccard similarity is used to estimate trust between users based on their rating history.
        \item These estimated trusts are then integrated into a classical recommender systems based on graphs.
    \end{itemize}

    \begin{equation}
        Jaccard(u,v) =  \frac{|I_u \cap I_v|}{|I_u \cup I_v|}
    \end{equation}
    with $I_u$ being the set of items that user $u$ purchased and $I_v$ being those that user $v$ purchased on the e-commerce site.\\
    \vspace{0.5cm}
    \textbf{Limits:} Trust here is non-asymmetrical and does not take time into account.
    
\end{enumerate}
\end{frame}



\section{Problem statement}
\begin{frame}{Problem statement}



\begin{block}{Problem statement}
Is it possible to take time into account when estimating social influence? 
\end{block}
\vspace{1cm}
The influence that one user has on another can change over time.\\
\vspace{1cm}
\textbf{We propose to take account of temporal dynamics when estimating the social influence between two users.}
\end{frame}





\section{Time aware social influence estimation}
\subsection{Estimation of social influence}

\begin{frame}{Architecture of contribution }
\begin{block}{\ }
    \begin{figure}
    \includegraphics[width=0.8\textwidth]{Architecture_rr.png}
    %\vspace{-0.5cm}
    \caption{Architecture of recommender system with time aware implicit social influence.} \label{fig:vc}
    \end{figure}
\end{block}
    
\end{frame}



\begin{frame}{Asymmetrical social influence }

\begin{itemize}
    % \item Jaccard similarity :
    % \begin{equation}
    %     Jaccard(u,v) =  \frac{|I_u \cap I_v|}{|I_u \cup I_v|}
    % \end{equation}
   \item Asymmetrical social influence (A):
   \begin{equation}
        \bm{InfsoA(u,v) =  \frac{|I_u \cap I_v|}{|I_u|}}
    \end{equation}

   \item Asymmetrical social influence + sequencing (S):
   \begin{equation}
    \bm{InfSoS(u,v) =  \frac{|u \rightarrow v|}{|I_u |}} 
    \end{equation}
\end{itemize}
\end{frame}

\begin{frame}{Temporal asymmetric social influence}
    \begin{itemize}
        \item Temporal asymmetric social influence (T)
        \begin{equation}
            \bm{InfSoT(u,v) = \frac{\sum_{|u \rightarrow v|}^{} f(t_u - t_v)}{|I_u |}}    
        \end{equation}

         \begin{figure}[ht]
            \centering
            \begin{minipage}{0.43\textwidth}
                \centering
                \includegraphics[width=\linewidth]{EDF.png}
                \caption{Exponential decay function - EDF  }
                \label{fig:edf}
            \end{minipage}\hfill
            \begin{minipage}{0.4\textwidth}
                \centering
                \includegraphics[width=\linewidth]{LDF.png}
                \caption{Logistic decay function - LDF }
                \label{fig:ldf}
            \end{minipage}\hfill
    
        \end{figure}
        %\includegraphics[height=0.37\textwidth]{MF.png}
        \quad $f(x) = e^{-x.ln(2)/to}$ \qquad  \qquad  \qquad \qquad \qquad \quad $f (x) = 1- 1/(e^ {-K(x-to)}+1)$
       
    \end{itemize}
    
\end{frame}

\subsection{Integration of social influence into KNN}
\begin{frame}{K nearest neighbours based recommender system }
    \begin{itemize}
        \item \textbf{K-nearest neighbour model based recommender system (KNN)}

        \begin{block}{Steps of the algorithm }
            \begin{enumerate}
                \item Calculation of the similarity between all users and the target user $u$
                    \begin{equation}
                    Sim(u,v) = Pearson(u,v) = \frac{ \sum_{i\in I_u\cap I_v} (r_{ui}-\mu_u).(r_{vi}-\mu_v) }{ \sqrt{\sum_{i\in I_u\cap I_v} (r_{ui}-\mu_u)^2} . \sqrt{\sum_{i\in I_u\cap I_v} (r_{vi}-\mu_v)^2} }
                    \end{equation}
                
                \item Select the K users who are most similar to $u$ and who rated item $i$. ($P_u(i)$)
                
                \item Predict the rating that $u$ will give to $i$.
                   \begin{equation}
                        r_{ui} = \mu_u + \frac{ \sum_{v\in P_u(i)} Sim(u,v).(r_{vi}-\mu_v) }{ \sum_{v\in P_u(i)} \lvert Sim(u,v) \rvert }
                   \end{equation}
            \end{enumerate}
            
        \end{block}
       
    \end{itemize}
    
\end{frame}


\begin{frame}{K nearest neighbours with social influence }
    \begin{itemize}
        \item \textbf{Integration of social influence into KNN}

        \begin{block}{Steps of the algorithm }
            \begin{enumerate}
                \item Calculation of the social influence on $u$ of other users $v$.

                $InfSo(u,v) \in \{InfSoA, InfSoS, InfSoT\}$
                    
                \item Select the top K influencers from $u$ who rated item $i$. ($P_u(i)$)
                
                \item Predict the rating that $u$ will give to $i$.
                   \begin{equation}
                        r_{ui} = \mu_u + \frac{ \sum_{v\in P_u(i)} InfSo(u,v).(r_{vi}-\mu_v) }{ \sum_{v\in P_u(i)} \lvert InfSo(u,v) \rvert }
                   \end{equation}
            \end{enumerate}
            
        \end{block}
       
    \end{itemize}
    
\end{frame}

\section{Experiments and Results}
\begin{frame}{Description and division of data sets}


\begin{block}{Description of data sets}
Set of tuples (u, i, c, r, t).

    \begin{table}
    \caption{Description of data sets Epinions and Ciao.}\label{tab:descrition_datasets}
    \vspace{-0.5cm}
    \resizebox{\textwidth}{!}{
    \begin{tabular}{|l|l|l|l|l|l|l|}
    \hline
    \textbf{Name} & \textbf{NbUsers} & \textbf{NbItems} & \textbf{minU} & \textbf{minI} & \textbf{NbRating} & \textbf{Period} \\
    \hline
    \textbf{Ciao} & 889 & 9053 & 1 & 1 & 12742 & 2007-2011  \\
    \hline
    \textbf{Epinions} & 728 & 18141 & 20 & 2 & 58717 & 2006-2010 \\
    \hline
    \end{tabular}
    }
    \end{table}
\end{block}

\begin{block}{Division of data sets}

    \begin{figure}
    \includegraphics[width=0.7\textwidth]{Validation_croisee.png}
    \vspace{-0.5cm}
    \caption{Cross-validation with increasing time window size.} \label{fig:vc}
    \end{figure}
    
\end{block}
\end{frame}

\begin{frame}{Evaluation metrics and parameter values}
    

\begin{block}{Evaluation metrics}
\begin{equation} \label{eq: rmse}
    RMSE = \sqrt{\frac{\sum_{u,i \in r_{test}} (r_{ui} - \hat{r}_{ui})^2 }{|r_{test}|}} 
\end{equation}

\begin{equation} \label{eq: mae}
    MAE = \frac{\sum_{u,i \in r_{test}} |r_{ui} - \hat{r}_{ui}| }{|r_{test}|} 
\end{equation}

\textbf{The closer it is to 0, the better the model performs.}

%with $r_{test}$ the test data set.
    
\end{block}

\begin{block}{Predefined parameter values}

    \begin{table}
    \caption{Predefined parameter values for KNN}\label{tab:val_predef_knn}
    \vspace{-0.5cm}
    \resizebox{\textwidth}{!}{
    \begin{tabular}{|l|l|l|}
    \hline
    & \textbf{Parameter description} & \textbf{Predefined values}\\
    \hline
    \textbf{K} & Number of neighbors & 2, 3, 5, 10, 20, 30 \\
    \hline
    \textbf{To} & Half-life of EDF and LDF functions & 30, 60, 120, 240, 360 days\\
    \hline
    \end{tabular}
    }
    \end{table}
    
\end{block}
    
\end{frame}

\begin{frame}{Results}
%\begin{block}{Résultats}

\begin{figure}
\vspace{-0.7cm}
\includegraphics[width=\textwidth]{ciao_epinions_sc_eng_v_r.png}
\vspace{-0.8cm}
\caption{KNN results with Ciao and Epinions} \label{fig:knn_sc}
\end{figure}
    
%\end{block}
\vspace{-0.3cm}
%\textbf{B:} Basic KNN \quad \textbf{J:} KNN with Jaccard similarity \quad
%\textbf{A:} KNN with asymetric social influence \quad \textbf{S:} KNN with asymetric + sequence sc \quad \textbf{TE:} KNN with asymetric temporal sc with EDF \quad \textbf{TL:} KNN with asymetric temporal sc with LDF

\textbf{B:} Basic KNN \qquad \qquad \qquad \qquad \quad \quad  \quad \textbf{J:} KNN with Jaccard similarity \\
\textbf{A:} KNN with asymetric social influence \hspace{0.1cm} \textbf{S:} KNN with asymetric + sequence sc \\
\textbf{TE:} KNN with temporal sc with EDF \quad \hspace{0.01cm} \textbf{TL:} KNN with temporal sc with LDF
    \vspace{-0.8cm}
\end{frame}

\begin{frame}{Best parameter values}

    \begin{itemize}
        \item \textbf{Number of neighbors (K):} 2 et 3 \\
        The top 2 or top 3 influencers of the target user are more predictive of the rating he will give to the items.
        \vspace{1cm}
        \item \textbf{Half-life (To) :} 30 days\\
        After 30 days, the influence that one user has on another with regard to an item is halved
    \end{itemize}
    
\end{frame}

\section{Conclusion and Futur work}
\begin{frame}{Conclusion and Futur work}
\begin{block}{Conclusion}

\begin{itemize}
    \item We looked at implicit trust by exploiting the history of ratings that users assign to items

    \item We have proposed a framework that takes time into account when estimating social influence.

    \item We obtained interesting results by integrating these social influences into the KNN-based recommender system.

    
    % \item Implicit social influence (rating history)
    % \item Time aware social influence \\
    % The influence one user has on another changes over time
    % \item Integration into KNN
\end{itemize}
    
\end{block}
\begin{block}{Futur work}
\begin{itemize}
    \item Take into account that the influence that one user has on another varies according to the category of item.
    
    \item Integrating social influence into matrix factorization based recommender systems  and graphs based recommender systems.
   
    \item Experiment with other data sets.
    %Pompare, Movilens, RetailRocket
    \item Evaluate with others metrics like NDGC, MAP, Hit Ratio.
\end{itemize}
    
\end{block}
    
\end{frame}

\begin{frame}{References}
\textbf{1.} Nzeko’o Armel Jacques Nzekon: A general graph-based framework for top-N recommendation using content, temporal and trust information. Journal of Interdisciplinary Methodologies and Issues in Sciences 5, (2019)\\
\textbf{2.} Mei Jian-Ping: A social influence based trust model for recommender systems. In-
telligent Data Analysis 21(2), 263–277 (2017) \\
\textbf{3.} Guo Guibing, Zhang Jie, Thalmann Daniel, Basu Anirban, Yorke-Smith Neil.: From
ratings to trust: an empirical study of implicit trust in recommender systems. In:
Proceedings of the 29th annual acm symposium on applied computing, pp. 248–253.
, (2014) \\
\textbf{4.}  Tang Jiliang, Gao Huiji, Liu Huan: mTrust: Discerning multi-faceted trust in a
connected world. In: Proceedings of the fifth ACM international conference on Web
search and data mining, pp. 93–102., (2012) \\
\textbf{5.} Ricci Francesco, Rokach Lior, Shapira Bracha: Recommender systems handbook.
Springer, 1–35 (2011)\\
\textbf{6.} Adomavicius Gediminas: Toward the next generation of recommender systems: A
survey of the state-of-the-art and possible extensions. IEEE transactions on knowl-
edge and data engineering 17(6), 734–749 (2005)
    
\end{frame}


















% \begin{frame}{Etat actuel des travaux }
%     \begin{itemize}
%         \item Implémentation du modèle de K nearest neighbors (KNN)
%         \begin{center}
            
%             \includegraphics[height=0.46\textwidth]{knn.png}    
%         \end{center}
        
       
%     \end{itemize}
    
% \end{frame}

% \begin{frame}{Etat actuel des travaux }
% Les fonctions de similarité :
%     \begin{itemize}
%         \item Similarité cosinus
%         \[ Sim(u,v) = cosine(u,v) = \frac{ \sum_{i\in I_u\cap I_v} (r_{ui}).(r_{vi}) }{ \sqrt{\sum_{i\in I_u\cap I_v} (r_{ui})^2} . \sqrt{\sum_{i\in I_u\cap I_v} (r_{vi})^2} }\]
        
%         \item Similarité de pearson
%         \[ Sim(u,v) = Pearson(u,v) = \frac{ \sum_{i\in I_u\cap I_v} (r_{ui}-\mu_u).(r_{vi}-\mu_v) }{ \sqrt{\sum_{i\in I_u\cap I_v} (r_{ui}-\mu_u)^2} . \sqrt{\sum_{i\in I_u\cap I_v} (r_{vi}-\mu_v)^2} }\]
%        \item Similarité de Jaccard
       
%        \[ Sim(u,v) = Jaccard(u,v) =  \frac{|I_u \cap I_v|}{|I_u \cup I_v|}\]
       
%     \end{itemize}
   
% \end{frame}

% \begin{frame}{Etat actuel des travaux }
%      La fonction de prédiction de notes 

%     \[r_{ui} = \mu_u + \frac{ \sum_{v\in P_u(i)} Sim(u,v).(r_{vi}-\mu_v) }{ \sum_{v\in P_u(i)} \lvert Sim(u,v) \rvert } \] \\

%     Notre idée pour le calcul des influences sociales s'inspire de la formule de la similarité de Jaccard précédente.
%     \[ Sim(u,v) = InfSo(u,v) =  \frac{|I_u \cap I_v|}{|I_u |} \]
%     Ici, $Sim(u,v)$ représente l'influence que subit $u$ par $v$. Cette nouvelle facon de calculer la similarité nous permettra de capturer au mieux l'influence sociale qu'exerce un utilisateur $v$ sur $u$.
% \end{frame}

% \begin{frame}{Etat actuel des travaux }
% \begin{itemize}
    

%  \item Jeux de données Epinions 
%   \includegraphics[height=0.6\textwidth]{Epinions.png}
   
%         \end{itemize}
    
% \end{frame}



% \begin{frame}{Etat actuel des travaux}

%     \begin{itemize}
    
%     \item Resultat de test 

%     \includegraphics[height=0.4\textwidth]{Results1.png}
   
%     \end{itemize}
    
% \end{frame}


\section*{Acknowledgement}  
\begin{frame}

\textcolor{myNewColorA}{\huge{\centerline{Thank you for your attention!}}}
\vspace*{0.5cm}

%{\Large{\centerline{E-mail: bleriot.tchamba@gmail.com}}}

\end{frame}



\end{document}



