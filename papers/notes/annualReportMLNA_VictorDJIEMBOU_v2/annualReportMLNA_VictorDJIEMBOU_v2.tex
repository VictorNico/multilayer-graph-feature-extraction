\documentclass[12pt,a4paper]{article}
\usepackage{pgf}
% \usepackage[condensed,math]{kurier}
\usepackage[T1]{fontenc}
\usepackage{svg}
%\usepackage[french]{babel}
\usepackage{tikz}
\usepackage{stanli}
\usepackage{afterpage}
\usepackage{multirow}
%\usepackage{subfig}
\usepackage{pgfpages}
\usepackage{pdfpages}
\usepackage{svg}
\usepackage{rotating}
\usepackage{multicol}
\usepackage[utf8]{inputenc}
\usepackage{translator}
\usepackage{float}
\usepackage{subfigure}
\usepackage{algorithm}
\usepackage{algpseudocode}
\usepackage{acronym}
% Language setting
% Replace `english' with e.g. `spanish' to change the document language
\usepackage[french]{babel}
% Useful packages
\usepackage{amsmath}
\usepackage{graphicx}
\usepackage{graphics}
\usepackage[colorlinks=true, allcolors=black]{hyperref}
\usepackage{tikz}
\usetikzlibrary{positioning}
% Set page size and margins
% Replace `letterpaper' with `a4paper' for UK/EU standard size
\usepackage[a4paper,top=2cm,bottom=1.5cm,left=1.5cm,right=1.5cm,marginparwidth=1.75cm]{geometry}
\usepackage{fancyhdr}
% modules
\usepackage{enumitem}
\usepackage{caption}

\usepackage{geometry}
\geometry{hmargin=2cm,vmargin=2cm}

\usepackage[bottom]{footmisc}
%\usepackage[color=True, allcoloring=blue]{hyperref}

%\usepackage{times}




% page numbering
%\pagestyle{fancy}
%\fancyhf{}
%\rhead{\thepage}
%\renewcommand{\headrulewidth}{0pt}


\title{}
\author{}
\date{}

%Sets the margins
%\textwidth = 7 in
%\textheight = 10 in
%\oddsidemargin = -0.4 in
%\evensidemargin = -0.5 in
%\topmargin = -0.4 in
%\headheight = 0.0 in
%\headsep = 0.0 in
%\parskip = 0.1in
%\footskip = 0.1 in 
%\parindent = 0.0in
% Define the boxed layout
%\pgfpagesdeclarelayout{boxed}
%{
%    \edef\pgfpageoptionborder{0pt}
%}
%{
%   \pgfpagesphysicalpageoptions
%    {%
%        logical pages=1,%
%    }
%    \pgfpageslogicalpageoptions{1}
%    {
%        border code={},%
%        %border shrink=\pgfpageoptionborder,%
%        resized width=\pgfphysicalwidth,%
%        resized height=\pgfphysicalheight,%
%        center=\pgfpoint{.5\pgfphysicalwidth}{.5\pgfphysicalheight}%
%    }%
%}

\fancypagestyle{firstpage}{
	\fancyhf{} % Clear header and footer
	\renewcommand{\headrulewidth}{0pt} % Remove header rule
	\renewcommand{\footrulewidth}{0pt} % Remove footer rule
	\fancyfoot[C]{} % Remove page number in the center
}


\newcounter{tableau}
\DeclareCaptionLabelFormat{figure}{Figure \thefigure}
\DeclareCaptionLabelFormat{tableau}{Tableau \thetableau}


\begin{document}
%\pgfpagesuselayout{boxed}
%\thispagestyle{firstpage}
\includepdf[pages=1, scale=1.1]{guard/guardPageReport.pdf}  % Include only page 1
\newpage
\tableofcontents

\pagebreak

% Définition des acronymes
\acrodef{IA}{Intelligence Artificielle}
\acrodef{AA}{Apprentissage Automatique}
\acrodef{MLN}{MultiLayer Network}
\acrodef{PME}{Petites et Moyennes Entreprises}
\acrodef{ANN}{Artificial Neural Network}
\acrodef{RNN}{Recurrent Neural Network}
\acrodef{LSTM}{Long Short-Term Memory}
\acrodef{DT}{Decision Tree}
\acrodef{LR}{Logistic Regression}
\acrodef{XGB}{eXtreme Gradient Boosting}
\acrodef{SVM}{Support Vector Machine}
\acrodef{RF}{Random Forest}
\acrodef{SCF}{Supply Chain Finance}
\acrodef{HARA}{Hub Authority Ranking Applicants Algorithm}
\acrodef{HubAvgRA}{Hub-Avg ranking applicants Algorithm}
\acrodef{ATkRA}{Authority-Threshold Algorithm}

\setlist[itemize]{label=\textbullet}

% introduction générale
\section{Introduction}
Les crédits bancaires sont l'une des sources importantes d'enrichissement des banques, cependant, ils peuvent également être source de grandes pertes financières. Afin de minimiser les pertes causées par les activités de prêts, des techniques avancés d'intelligence artificielle sont proposées spécialement pour la prédiction du risque de crédit bancaire \cite{ince2009comparison, trivedi2020study, nalic2020new}. 

La plupart des travaux sur la prédiction du risque de crédit considèrent uniquement les attributs de description des prêts (caractéristiques de l'emprunteur et du prêt), et ne s'attardent pas particulièrement sur la modélisation explicite des relations entre les emprunteurs ou même entre les emprunteurs et d'autres caractéristiques considérés dans les données. Ceci peut être une limite, car les individus aux caractéristiques communes peuvent avoir les mêmes comportements de prêts, et donc à partir des comportements connus d'un ensemble d'individus similaires à un individu cible, on peut déduire le comportement de ce dernier.

Dans cette section du projet, il est question de modéliser les données d'une base de prêts bancaires par des graphes dont la définition des nœuds et des arcs est suffisamment pertinente pour que les nouveaux descripteurs extraits de ces graphes contribuent fortement à la décision des modèles de classification supervisée pour la prédiction du risque de crédit. L'approche adoptée débute par la construction d'un graphe à partir des données bancaires, se poursuit par l'application des algorithmes de calcul des descripteurs sur des graphes, et se termine par l'ajout des nouveaux descripteurs aux données d'entrée des algorithmes de prédiction du risque de crédit. Plus spécifiquement, nous étudions en profondeur l'extraction des descripteurs du graphe multicouches MLN (\textit{MultiLayer Network}) pour la prédiction du risque de crédit \cite{oskarsdottir2021multilayer}. 

Le reste du document est structuré comme suit : nous débutons par la section~\ref{sec:generality} pour la présentation des généralités sur les graphes et les descripteurs extraits des graphes. Ensuite, nous présentons dans la section~\ref{sec:mlna} la modélisation par des graphes multicouches et les descripteurs extraites de ces graphes. Enfin, nous présentons quelques expérimentations réalisées et les résultats obtenus à la section~\ref{sec:exp} et nous terminnons par une conclusion sur le travail fournis et quelques perspectives dans la section~\ref{sec:con}.
% Généralité sur les descripteurs
\section{Descripteurs extraits des graphes}
\label{sec:generality}

\subsection{Généralité et définition}
%Un graphe $G$ est une paire d'ensemble $G = (V,E)$. $V$ est l'ensemble de noeuds et le nombre de noeuds $n = \lvert V \rvert$ désigne l'ordre du graphe. L'ensemble $E$ contient les arêtes du graphes. Dans un graphe non-orienté, chaque arête est une pair non ordonnée $\{v,w\}$. Dans un graphe orienté, les pairs sont ordonnées. Les noeuds $v$ et $w$ sont appelés terminaux et le nombre d'arêtes $m = \lvert E \rvert$ représente la taille du graphe. Dans un graphe pondéré, $W: E \to R$ est une fonction qui permet d'assigner un point à chaque noeud du graphe. La densité d'un graphe $G = (V,E)$. $V$ est le ratio entre le nombre d'arêtes dans le graphe et le nombre d'arêtes maximal possible, $\delta(G) = \frac{m}{n}$ tel que pour $n\in [0,1]$ alors $\delta(G) = 0$. Ainsi une graphe de densité 1 sera appelé graphe complet. Un graphe biparti est celui la où l'ensemble de noeuds $V$ peut être décomposé en deux groupe $A$ et $B$ tel que $\{v,w\} \in E$ si ($v \in A$ et $w \in B$) ou ($v \in B$ et $w \in A$).
Un graphe $G = \{V,E\}$ est une structure de données qui permet de modéliser les relations entre des entités. La modélisation d’un graphe repose sur deux notions, celle de nœuds $V$ et celle d’arcs $E$. La notion de nœud est associée aux entités qui sont en relation et celle d’arc est associée à la nature de la relation d’une entité (nœud) avec une autre.

Après la construction d’un graphe, il est possible d’extraire des variétés de descripteurs liées soit aux nœuds, soit à la relation entre les nœuds, et même à la topologie du graphe.

\begin{itemize}
    \item \textbf{Descripteurs d’un nœud :} on peut citer les mesures de centralité qui estiment à quel point le nœud est incontournable dans la navigation dans le graphe. C’est le cas par exemple du PageRank\cite{pastor2015epidemic} où initialement on attribue le même poids à chaque nœud, puis chaque nœud diffuse son poids à tous ses voisins directs proportionnellement aux poids des relations avec ses voisins. Le processus est répété jusqu'à ce que les poids des nœuds ne changent plus, ou alors jusqu'à ce qu’un nombre maximum d’étapes de diffusion soit atteint. Les nœuds aux poids les plus grands, sont les plus importants.
	
    \item \textbf{Descripteurs d’une relation entre deux nœuds :} on peut parler des mesures qui décrivent la relation entre deux nœuds à partir du nombre et de la longueur des plus courts chemins entre ces nœuds.
	
    \item \textbf{Descripteurs de la topologie du graphe :} il est possible d’extraire des communautés dans un graphe (sous-ensemble de nœuds densément connectés entre eux et faiblement connectés au reste du graphe).
\end{itemize}

Le procédé qui consiste à construire un graphe et à calculer des descripteurs, permet d’apporter de nouvelles informations pour enrichir la description des entités considérées.

%Les noeuds les plus important ou influant dans un graphes ne sont pas toujours ceux qui ont une forte connexion (les hubs) mais plutôt ceux qui ont une stratégiquement positionnés dans le réseau\cite{vega2019multi,pastor2015epidemic}. Alors certains mesures dans l'analyse de graphe se propose de proposer un ordre d'importance entre les noeuds sur la base de leur influence\cite{pastor2015epidemic}. 

%Parmi tant d'autre nous avons, la centralité de degrée vient mesuré la connectivité d'un noeud, plus elle est élevée plus le noeud est considéré comme centrale en terme de connectivité au réseau. Cette mesure demeure local à un noeud dans le graphe. Il existe néanmoins des mesure avec un porté globale dans le graphe. 

%La centralité de proximité mesure la distance moyenne entre un noeud et tous les autres dans le réseau, et ceux qui plus proche des autres noeuds du réseau sont considéré comme centraux. La centralité d'intermediairité dans quelle mesure un noeud se trouve sur les chemins les plus courts entre les autres noeuds du graphe. Les noeuds qui se trouvent sur de nombreux chemins les plus courts sont considérés comme centraux en termes de contrôle de l'information ou de médiation entre les autres noeuds.  

%La centralité de vecteur propre (Eigenvector centrality) attribue une importance aux noeuds en fonction de l'importance des noeuds auxquels ils sont connectés. Les noeuds qui sont liés à d'autres noeuds importants ont une centralité plus élevée. Le PageRank évalue l'importance des pages web dans un graphe du web. Il attribue une importance aux nœuds en fonction du nombre et de l'importance des liens entrants.

\subsection{Travaux existants sur l'extraction des descripteurs des graphes}
Dans la littérature sur la prédiction du risque de crédit, quelques travaux procèdent à des modélisations des données de prêts par des graphes. A l'issu de la construction du graphe, des opérations y sont appliquées pour extraire de nouveaux descripteurs qui peuvent être pertinents.

Mario et al. \cite{mario2021graph} capturent les relations financières entre entités intervenantes (emprunteurs, instituts financiers) dans le contexte d’une inter-coopération des institutions financières et des emprunteurs pour modéliser un graphe orienté représentant une micro-structure du réseau où chaque noeud représente un emprunteur ou une institution financière, et les arêtes, le lien financier existant entre les noeuds. Les auteurs vont extraire des mesures telles que le dégrée d'un noeud, les longueurs des plus courts chemins, le sous graphe maximal de distance minimale afin d’améliorer la prédiction du risque de crédit avec des modèles comme SVM -\textit{\ac{SVM}} et la Régression Logistique -\textit{\ac{LR}}.

Xu et al. \cite{xu2009credit} construisent un graphe biparti pour représenter les informations d'historique des prêts dans les institutions bancaires. Ici chaque noeud matérialise soit un emprunteur soit une modalité des attributs, et une arrête entre un noeud emprunteur et un noeud modalité signifie que l'emprunteur est décrit par cette modalité. Pour prendre en compte les relations complexe entre les emprunteurs, ils vont utiliser trois algorithmes d'analyse de liens basés sur \ac{HARA}, \ac{HubAvgRA}, \ac{ATkRA} pour extraire des descripteurs qui sont positionnés en entrée de l'apprentissage du modèle SVM.

Giudici et al. \cite{giudici2020network} procèdent à la construction des graphes complets dans lesquels un noeud est un prêt ou un emprunteur, et les liens ou arêtes entre les noeuds sont pondérés par des valeurs qui correspondent à la similarité entre les deux noeuds associés. Une fois le graphe complet construit, plusieurs algorithmes d'extraction des caractéristiques sur des graphes sont appliqués (mesures de centralité des noeuds), et les résultats obtenus sont positionnés en entrée du processus d'apprentissage des modèles de prédiction comme nouveaux descripteurs.  

Les graphes bipartis et les graphes complets ne sont pas les seuls types de graphes qui peuvent être exploités pour la modélisation des données bancaires. Ainsi, explorer d'autres modélisation en graphe dans le but d'extraire des attributs pertinents reste une voie intéressante pour les futurs travaux sur les descripteurs extraits des graphes. Une des pistes envisagée récemment est celle de la modélisation des données bancaires par des graphes multicouches afin de mieux capturer les relations entre emprunteurs, et entre emprunteur et leurs caractéristiques.

%Ses deux approches font bien d'utiliser des techniques d'analyse de graphes pour capturer des descripteurs pertinents pouvant améliorer la prédiction du risque de crédit par des algorithmes d'apprentissage automatique, toutefois, elles ne considèrent pas que la relation entre emprunteurs peut être fortement influencée par un sous ensemble de caractéristiques ou niveaux. Hors hypothétiquement, si je suis en relation avec un individu c'est forcement au moins à cause d'un niveau de caractérisation en commun.

% Descripteurs extraits des graphes multicouches
\section{Descripteurs extraits des graphes multicouches}
\label{sec:mlna}
La lecture de l’article « Multilayer network analysis for improved credit risk prediction » de Mar{\'i}a {\'O}skarsd{\'o}ttir et Cristi{\'a}n Bravo \cite{oskarsdottir2021multilayer}, est le point d’entrée pour l’usage des descripteurs issus de la modélisation par un graphe multicouches, avec pour objectif d'enrichir les données d’apprentissage des modèles classiques de prédiction du risque de crédit.

En effet, les auteurs utilisent la modélisation par des graphes multicouches (multilayer network) où un emprunteur a autant de nœuds qu’il y a de dimensions qui le caractérisent, et dans chaque dimension, il est relié à des attributs qui le définissent suivant cette dimension. Ainsi, plus les emprunteurs sont similaires, plus ils sont proches dans chacune des dimensions du graphe multicouches. Une dimension peut par exemple être la localisation géographique ou encore le type d’activité exercé comme c'est le cas dans les travaux de {\'O}skarsd{\'o}ttir et al. \cite{oskarsdottir2021multilayer}.

L'une des hypothèses sur laquelle les auteurs s’appuient, est celle selon laquelle, les emprunteurs qui ont un grand nombre de caractéristiques en commun (dans les différentes dimensions) ont une grandes probabilités d’avoir des prêts de la même classe (PAYE ou IMPAYE).

Ainsi, il se pose les difficultés suivantes :
\begin{itemize}
\item Comment établir les relations entre les emprunteurs ?
\item Comment déduire des caractéristiques à exploiter à partir de la nouvelle représentation ?
\item Comment prédire la classe d’un prêt ?
\end{itemize}

\subsection{Processus de construction du graphe}
Pour établir les relations entre les emprunteurs, les auteurs proposent de construire un graphe multicouches, et pour ce faire, on fixe les dimensions considérées et les attributs associés à chacune de ces dimensions. Dans le cas des travaux de {\'O}skarsd{\'o}ttir et al. \cite{oskarsdottir2021multilayer} il est question de prêts agricoles et seules deux dimensions sont choisies : la localisation géographique et les produits vendus par les agriculteurs. Les attributs de la dimension localisation géographique peuvent être le district, l’arrondissement etc, et concernant la dimension produit, les attributs peuvent être les différents produits répertoriés.

La construction du graphe multicouches obéit aux règles suivantes :
\begin{itemize}
	\item Chaque emprunteur a autant de nœuds qu’il y a de couches considérées
	\item Les nœuds de chaque emprunteur sont tous reliés les uns aux autres
	\item Chaque attribut d’une dimension a un nœud associé
	\item Si un emprunteur est décrit par un attribut dans une dimension donnée, alors le nœud emprunteur de cette dimension est relié au nœud attribut associé
	\item La navigation d’une couche à une autre se fait en passant par les nœuds emprunteurs des différentes couches
\end{itemize}

Dans l’article, les auteurs considèrent deux dimensions pour décrire les emprunteurs dans le graphe multicouches, à savoir la localité et les produits vendus par ces derniers. Considérons un cas où nous avons 4 emprunteurs (des fermiers), 2 localités et 3 produits agricoles vendus par les fermiers. Dans ce cas, nous avons 3 couches (Emprunteur, Localité et Produit), et nous avons 9 nœuds (4 nœuds emprunteurs + 2 nœuds localités + 3 nœuds produits), et donc la matrice carrée qui permet de représenter le graphe multicouches est de taille (9 x 3) x (9 x 3).

\newpage
\begin{figure}[h]
  \centering
  \includegraphics[width=0.7\textwidth]{assets/mln}
  \captionsetup{labelformat=figure}
  \caption{A gauche on a un exemple de graphe multicouches à deux couches, et à droite on a la représentation de ce graphe sous forme de matrice. Ce graphe contient 9 emprunteurs (nœuds noirs), 3 nœuds de la dimension localité des emprunteurs (nœuds marron) et 3 nœuds de la dimension produits (vert, rouge, jaune). Les relations inter- couches sont matérialisées par des traits interrompus et existent uniquement entre les nœuds emprunteurs qui représentent le même emprunteur dans les différentes couches. Les relations intra-couche sont matérialisées par les autres types de trait (trait marron dans la dimension localité et les autres couleurs dans la dimension produit).}
  \label{fig:mln}
\end{figure}
%\begin{figure}
%  \centering
%  \includegraphics[width=0.9\textwidth]{assets/mln_1}
%  \captionsetup{labelformat=figure}
%  \caption{Représente le même graphe multicouches précédent, mais avec des détails supplémentaires
%sur le procédé de construction du graphe et de la matrice associée. Un emprunteur (fermier) est relié à sa localité et aux produits agricoles qu’il commercialise.}
%  \label{fig:mln1}
%\end{figure}

\subsection{Extraction des nouveaux descripteurs du graphe multicouches}
Lorsque le graphe multicouches est construit, les nouveaux descripteurs du prêt sont calculés suite à des applications du PageRank Personnalisé sur le graphe résultat. Ceci est possible car un graphe multicouches M, ayant N nœuds, et L couches, correspond à une représentation de dimension N x N x L x L, ce qui peut être résumé en une matrice carrée (N x L) x (N x L) sur laquelle ont peut appliquer le PageRank. Ainsi, les auteurs proposent 03 façons différentes de calculer les nouveaux descripteurs :

\begin{itemize}
    \item \textbf{Intra-influence :} le PageRank Personnalisé est initialisé de manière à favoriser les relations intra-couche dans le processus de diffusion.
    \item \textbf{Inter-influence :} le PageRank Personnalisé est initialisé de manière à favoriser les relations inter-couche dans le processus de diffusion.
    \item \textbf{Influence-combinée :} le PageRank Personnalisé ne favorise pas un type de relation.
\end{itemize}

\begin{figure}[!h]
  \centering
  \includegraphics[width=0.9\textwidth]{assets/global}
  \captionsetup{labelformat=figure}
  \caption{Les trois scénarios considérés par les auteurs pour calculer les nouveaux descripteurs des prêts par l’application du PageRank Personnalisé sur le graphe multicouches.}
  \label{fig:mln1}
\end{figure}

Notons que d’autres descripteurs sont considérés dans le graphe multicouches :
\begin{itemize}
	\item Nombre de nœuds emprunteurs qui vendent les mêmes produits que l’emprunteur cible
	\item Nombre de nœuds emprunteurs défaillants qui vendent les mêmes produits que l’emprunteur cible
	\item Nombre de nœuds emprunteurs de la même localité que l’emprunteur cible
	\item Nombre de nœuds emprunteurs défaillants de la même localité que l’emprunteur cible
	\item Nombre d’emprunteurs de la même localité que l’emprunteur cible et qui vendent les mêmes produits que lui
	\item Nombre d’emprunteurs défaillants de la même localité que l’emprunteur cible et qui vendent les mêmes produits que lui
\end{itemize}

\subsection{Prédiction du risque de crédit avec les nouveaux descripteurs}
Pour procéder à la prédiction du risque de crédit avec les modèles classiques d’apprentissage automatique, les descripteurs présents dans le jeu de données, et les nouveaux descripteurs extraits des graphes, sont utilisés comme données d’apprentissage des modèles classiques choisis (Régression logistique et XGBoost) pour la prédiction des risques de crédit.

Une fois que ces modèles sont construits, ces derniers sont utilisés pour prédire les classes des prêts du jeu de test. Dans l’article, les comparaisons des performances des modèles avant et après l’insertion des nouveaux descripteurs, montrent que les nouveaux descripteurs améliorent la qualité des prédictions. Par ailleurs, les analyses sur l’explicabilité de ces modèles ont montré que les nouveaux descripteurs étaient parmi ceux qui contribuent le plus à la prise de décision des modèles de Régression logistique et XGBoost.

\subsection{Limites et perspectives au travail sur le graphe multicouches}
{\'O}skarsd{\'o}ttir et al. \cite{oskarsdottir2021multilayer} proposent une modélisation des données des prêts en graphe multicouches. Cette approche a pour fort de s'apparenter à la réalité de la vie mais suscite encore des critiques qui définissent en réalité des limites á leur modélisation.
\begin{itemize}
	\item Tous les attributs ne sont pas considérés dans le graphe multicouches.
	\item Le choix des attributs catégoriels à considérer comme couche du graphe multicouches est fait de façon arbitraire. Il serait intéressant de proposer un protocole pour ce choix.
	\item Les applications du PageRank Personnalisé sur le graphe multicouches sont globales à tout le graphe, et donc tous les noeuds sont traités pareil quelque soit le prêt pour lequel on veut faire une prédiction. Et pourtant, il serait intéressant d'avoir une personnalisation du PageRank qui favorise les noeuds qui sont liés ou qui sont proches de l'emprunteur dans le graphe multicouches. Par contre, on peut avoir des applications du PageRank personnalisé pour chacun des prêts ou emprunteur. 
\end{itemize}

\paragraph{Extensions possibles :} Ses critiques permettent d'émettre les extensions possibles suivantes :
 \begin{itemize}
    %\item Exploiter un graphe qui prend en compte tous les attributs descriptifs des prêts. On peut par exemple avoir un nœud pour chacune des modalités possibles de chaque attribut. Ensuite, relier tous les nœuds du même prêt ou alors incrémenter les poids des arcs qui relient tous les nœuds des modalités d’un prêt. Ensuite, appliquer le PageRank Personnalisé par chaque prêt, sur le graphe résultat, afin de ressortir avec de nouveaux descripteurs du prêt à l’exemple de l’estimer la classe de ce prêt par le PageRank.
    \item Proposer un graphe multicouches avec autant de couches que d'attributs catégoriels.
    \item Intégrer les attributs quantitatifs comme des couches du graphe multicouches.
    \item Proposer un protocole qui permet de choisir efficacement les attributs catégoriels à considérer dans le graphe multicouches.
    \item Personnaliser l'exécution du PageRank pour chaque prêt ou emprunteur au lieu d'une exécution pour tous les prêts et emprunteurs à la fois.
    \item Travailler avec plusieurs jeux de données.
    \item Intégrer la classe du prêt comme une couche du graphe multicouches.
\end{itemize}


% Expérimentations et résultats
\section{Expérimentation et résultats}
\label{sec:exp}
\subsection{Données}
Les détails des données manipulées lors de cette expérimentation sont données par le tableau suivant:

\begin{table}[H]
    \centering
    \resizebox{\textwidth}{!}{
    \begin{tabular}{cccccc}
         & Nb exemples & Nb attributs numériques & Nb attributs catégoriels & Nb payés & Nb impayés\\
        AFB\footnote{Données provenant d'un particulier.} & 28952 & 08 & 04 & 21769 & 7183 \\
        AER\footnote{\url{https://www.kaggle.com/dansbecker/aer-credit-card-data}} & 1319 & 08 & 02 & 1023 & 296 \\
        CREDIT RISK\footnote{} & 32581 & 07 & 03 & 25473 & 7108 \\
        GERMAN\footnote{\url{http://archive.ics.uci.edu/dataset/144/statlog+german+credit+data}} & 1000 & 07 & 13 & 700 & 300\\
        JAPAN\footnote{\url{http://archive.ics.uci.edu/dataset/28/japanese+credit+screening}} & 690 & 06 & 09 & 307 & 383 \\
    \end{tabular}
    }
    \caption{Description des données {\tiny \url{https://github.com/JLZml/Credit-Scoring-Data-Sets}}}
    \label{tab:my_label}
\end{table}
\subsection{Application des descripteurs du graphe multicouches sur le jeu de données AFB (Afriland First Bank)}
Pour expérimenter les concepts appris, il est nécessaire de choisir des attributs qualitatifs qui vont représenter les dimensions (couches) du graphe multicouches. Nous avons donc recensé les attributs catégoriels du jeu de données AFB de Afriland First Bank.

Lorsqu'on ignore la classe des prêts, les attributs catégoriels de ce jeu de données sont :
\begin{itemize}
	\item \textbf{Type / Motif : }le type de prêt ou motif du prêt bancaire
	\item \textbf{Fonction :} le métier ou l’occupation de l’emprunteur
	\item \textbf{Civilité :} civilité de l’emprunteur (Monsieur, Madame, Mademoiselle)
	\item \textbf{Statut matrimonial : }statut matrimonial de l’emprunteur (Célibataire, Marié, Divorcé)
\end{itemize}

Après avoir recensé les attributs catégoriels, nous avons choisi d'implémenter trois graphes multicouches à deux couches. Le premier graphe multicouches nommé ici \textbf{MLN1} est construit à partir des attributs \textbf{Fonction \& Civilité}, le second graphe \textbf{MLN2} est construit à partir des attributs \textbf{Fonction \& Statut-Matrimonial} et enfin, le troisième graphe \textbf{MLN3} est construit à partir des attributs \textbf{Fonction \& Motif}.

Les attributs extraits des ces différents graphes multicouches sont énumérés comme suit pour une paire d'attribut (Att1, Att2):
%\begin{itemize}
%\subsubsection{Attributs du MLN1 - Fonction \& Civilité}
	%\item \textbf{Attributs du MLN1 - Fonction \& Civilité}
	\begin{itemize}
		\item \textbf{MLN\_Att1\_degré :} le nombre d'emprunteurs avec la même valeur d'Att
		\item \textbf{MLN\_Att2\_degré :} le nombre d'emprunteurs avec la même valeur d'Att
		\item \textbf{MLN\_Att1\_et\_Att2\_degré :} le nombre d'emprunteurs qui ont des valeurs égales pour chacun des attributs Att1 et Att2 
		\item \textbf{MLN\_bipart\_intra\_Att1\_Att2 :} le score PageRank maximal entre le noeud de l'emprunt de couche Att1 et Att2 lorsque seul les noeuds intra (modalités Att1 et Att2) sont inclus dans le vecteur de personnalisation du PageRank
		\item \textbf{MLN\_bipart\_inter\_Att1\_Att2 :} le score PageRank maximal entre le noeud de l'emprunt de couche Att1 et Att2 lorsque seul les noeuds inter (emprunt ou emprunteur) sont inclus dans le vecteur de personnalisation du PageRank
		\item \textbf{MLN\_bipart\_combine\_Att1\_Att2 :} le score PageRank maximal entre le noeud de l'emprunt de couche Att1 et Att2 
		\item \textbf{MLN\_bipart\_intra\_Att1\_max :} le score PageRank maximal du noeud de Att1 associé à un emprunt lorsque seul les noeuds intra (modalités de Att1) sont inclus dans le vecteur de personnalisation du PageRank
		\item \textbf{MLN\_bipart\_inter\_Att1\_max :} le score PageRank maximal de Att1 associé à un emprunt lorsque seul les noeuds inter (emprunt ou emprunteur) sont inclus dans le vecteur de personnalisation du PageRank
		\item \textbf{MLN\_bipart\_combine\_Att1\_max :} le score PageRank maximal du noeud de Att1 associé à un emprunt
		\item \textbf{MLN\_bipart\_intra\_Att2\_max :} le score PageRank maximal du noeud de Att2 associé à un emprunt lorsque seul les noeuds inter (modalités de Att2) sont inclus dans le vecteur de personnalisation du PageRank
		\item \textbf{MLN\_bipart\_inter\_Att2\_max :} le score PageRank maximal du noeud de Att2 associé à un emprunt lorsque seul les noeuds inter (emprunt ou emprunteur) sont inclus dans le vecteur de personnalisation du PageRank
		\item \textbf{MLN\_bipart\_combine\_Att2\_max :} le score PageRank maximal du noeud de Att2 associé à un emprunt
	\end{itemize}

\subsubsection{Mise en œuvre : intégration des attributs extraits des graphes multicouches dans le processus de prédiction du risque de crédit}
Nous avons considéré cinq modèles classiques de l’apprentissage automatique pour la prédiction du risque de crédit : \ac{DT}, \ac{RF}, \ac{LR}, \ac{XGB} et \ac{SVM}. Pour chacun des graphes multicouches considérés (MLN1, MLN2, et MLN3), chaque modèle de prédiction est appliqué 04 fois. Chacune des applications du modèle diffère de l’autre par l’ensemble d’attributs descripteurs considérés :
\begin{itemize}
	\item \textbf{Classic} : les attributs considérés sont tous ceux fournis avec le jeu de données
	\item \textbf{Classic + MLN} : on considère tous les attributs du jeu de données et on ajoute les
autres attributs extraits du graphe multicouches \ac{MLN}
	\item \textbf{Classic – Att} : on considère une partie des attributs fournis avec le jeu de données. Ceux qui sont liés aux dimensions du graphe multicouches \ac{MLN} sont ignorés. Par exemple, pour le cas MLN1, les attributs relatifs à Fonction et à Civilités seront complètement écartés de la phase d’apprentissage
	\item \textbf{Classic + MLN - Att} : on écarte les attributs relatifs aux dimensions du graphe multicouches MLN, et on intègre les attributs extraits du graphe multicouches \ac{MLN}
\end{itemize}

Nous pouvons ainsi évaluer l’impact des attributs choisis dans le graphe multicouches associés à leur représentation standard fourni dans le jeu de données \textbf{(Classic + MLN)}, sans leur représentation standard \textbf{(Classic + MLN - Att)}, et enfin évaluer l’impact de leur absence des données d’apprentissage des modèles \textbf{(Classic – Att)}.

Le tableau \ref{fig:afbR} ci-dessous présente l’ensemble des résultats obtenus pour les cinq différents modèles de prédiction SVM, XGBoost, Arbre de décision, Régression Logistique et Forêt Aléatoire, avec les trois graphes multicouches considérés (MLN1, MLN2, MLN3) et suivant les métriques Exactitude, Précision, Rappel et F1-score

\begin{table}[!h]
  \centering
  \includegraphics[width=\textwidth]{assets/tableau_graph_page10}
  \caption{Résultats avec les différents graphes multicouches. Dans ce tableau, AMLNi correspond à Att et donc aux attributs considérés pour construire le graphe multicouches.}
  \label{fig:afbR}
\end{table}

Lorsqu’on s’attarde sur le classement des modèles de prédiction, le modèle de prédiction associé à la meilleure performance est la Forêt Aléatoire (Random Forest), suivi d’Arbre de décision et de \ac{XGB}. La Régression Logistique et SVM ferme ce classement. En s’intéressant aux meilleures performances obtenues avec les attributs extraits des graphes multicouches, on constate que le graphe multicouches MLN1 est le meilleur pour le modèles XGBoost, le graphe MLN2 est meilleur pour SVM et Arbre de décision et enfin le graphe MLN3 est le meilleur pour Régression Logistique et Forêt Aléatoire.

Si on considère uniquement les cas de figure où chaque modèle est associé au graphe multicouches qui lui correspond le mieux, on fait les remarques suivantes :
\begin{itemize}
\item  \textbf{SVM} : il ne faut pas considérer les attributs issus du graphe multicouches
\item  \textbf{XGBoost} : les meilleures performances sont atteintes lorsqu’on considère à la fois les attributs extraits du graphe multicouches ainsi que leur forme classique dans le jeu de données. Et la forme classique de ces attributs a plus d’impact que les attributs extraits du graphe multicouches.
\item  \textbf{Arbre de décision}: les meilleures performances sont atteintes lorsqu’on considère uniquement les attributs issus du graphe multicouches et qu’on ignore ces attributs dans leur représentation classique.
\item  \textbf{Régression Logistique} : les meilleures performances sont atteintes lorsqu’on considère uniquement les attributs issus du graphe multicouches et qu’on ignore ces attributs dans leur représentation classique.
\item  \textbf{Forêt Aléatoire} : en considérant la Précision comme métrique d’évaluation, la meilleure performance est atteinte lorsqu’on considère uniquement les attributs issus du graphe multicouches et qu’on ignore ces attributs dans leur représentation classique.
\end{itemize}

D’après les résultats obtenus, la considération des attributs extraits des graphes multicouches (Classic + MLNi et Classic – AMLNi + MLNi) permet l’amélioration des performances des modèles Arbre de décision, XGBoost, Random Forest et Régression Logistique. L’unique modèle pour lequel il n’y a pas d’amélioration mais un statuquo c’est le modèle SVM.

\paragraph{Évaluation de la contribution des attributs pour les modèles de prédiction :}
Suite à l’appréciation des résultats des modèles de prédiction, nous nous sommes intéressés aux contributions des différents attributs pour l’obtention des résultats considérés, notamment pour les cas de figure associés aux meilleurs graphes multicouches. Les graphiques des cinq pages suivantes illustrent ces informations (une page par modèle de prédiction).

Sur chaque graphique, plus une barre est longue et orientée vers la droite, plus l’attribut associé à cette barre contribue positivement aux décisions du modèle de prédiction. Par contre plus une barre est étirée vers la gauche, plus l’attribut contribue négativement aux décisions du modèle de prédiction.
Sur les graphiques, les barres peuvent avoir trois couleurs possibles :
\begin{itemize}
\item \textbf{Les barres bleus} ; correspondent aux attributs classiques fournis dans le jeu de données, mais qui sont différents des attributs considérés pour le graphe multicouches.
\item \textbf{Les barres jaunes} : correspondent aux attributs classiques fournis dans le jeu de données et qui sont considérés pour la construction du graphe multicouches.
\item \textbf{Les barres vertes}: correspondent aux nouveaux descripteurs issus du graphe multicouches correspondant.
\end{itemize}

En observant les graphiques des résultats des combinaisons de type Classic + MLNi, on constate que pour les modèles de prédiction Forêt Aléatoire (MLN\_bipart\_intra\_Motif\_max), XGboost (MLN\_Fonction\_et\_Civilité\_Degrée) et Arbre de Décision (MLN\_bipart\_intra\_\\ Fonction\_Sit\_Matrim), les descripteurs issus du graphe multicouches sont mieux classés que les attributs associés dans leur forme classique provenant du jeu de données. Par contre, la tendance est inversée pour les cas de SVM et la Régression Logistique où les descripteurs classiques sont mieux classés que ceux extraits des graphes multicouches.

Ce constat renforce la pertinence des descripteurs extraits des graphes multicouches, car ces derniers contribuent beaucoup plus pour les modèles de prédiction associés aux plus grandes performances. Ce qui est davantage renforcé par le cas de la Forêt Aléatoire qui est le modèle le plus performant de tous, et dont l’attribut MLN\_bipart\_intra\_Motif\_max extrait des graphes multicouches est celui qui contribue le plus aux prises de décisions.

\newpage

\begin{figure}[H]
  \centering
  \includegraphics[width=1.06\textwidth]{assets/sv_afb_shapley.png}
  \caption{SVM – meilleur graphe multicouches MLN2}
\end{figure}

\begin{figure}[H]
  \centering
  \includegraphics[width=1.06\textwidth]{assets/xgb_afb_shapley.png}
  \caption{XGBoost – meilleur graphe multicouches MLN1}
\end{figure}

\begin{figure}[H]
  \centering
  \includegraphics[width=1.06\textwidth]{assets/dtc_afb_shapley.png}
  \caption{Arbre de Décision (Decision Tree) – meilleur graphe multicouches MLN2}
\end{figure}

\begin{figure}[H]
  \centering
  \includegraphics[width=1.06\textwidth]{assets/lrc_afb_shapley.png}
  \caption{Régression Logistique – meilleur graphe multicouches MLN3}
\end{figure}

\begin{figure}[H]
  \centering
  \includegraphics[width=1.06\textwidth]{assets/rfc_afb_shapley.png}
  \caption{Forêt Aléatoire (Random Forest) – meilleur graphe multicouches MLN3}
\end{figure}


\subsection{Application du PageRank personnalisé pour chaque prêt}
\label{sec:pp}
Nous proposons dans cette expérimentation qu'un score de PageRank soit calculé pour chaque emprunts de tel sorte que pour un emprunt précis, l'exécution du PageRank permettant d'extraire ce score n'aille comme unique noeuds possible de démarrage initiale ceux associés à cet emprunt. Nous espérons par cette ultra personnalisation de l'extraction d'influence dans le réseau formé proposer des descripteurs pertinentes dans l'evaluation du risque de crédit. 

L'équation de calcul du PageRank personnalisé à un seul emprunteur est donnée par:

	\begin{align}
    		PR_{i+1}^U &= (1-\alpha) \times d + \alpha \times PR_{i}^U \times H, \\
    		&\text{sc.} \quad |d| = L \times N + M, \quad \sum_{i=0}^{|d|} d[i] = 1,
		 \quad d[i] =  0 \iff d[i] \notin U 
	\end{align}

	où :
	\begin{align*}
    		& PR_{i}^U \text{ est le PageRank à l'itération i} , \\
		& PR_{i+1}^U \text{ est le PageRank à l'itération i+1} , \\
    		& \alpha \text{ est le facteur d'amortissement (typiquement } \alpha = 0,85), \\
    		& N \text{ est le nombre total d'emprunts dans le graphe} \\
    		& H \text{ est la matrice de transition du graphe }.\\
    		& d \text{ est le vecteur de probabilités de démarrage}\\
		& L \text{ est le nombre de couches}\\
		& N \text{ est le nombre d'emprunts dans le graphe}\\
		& M \text{ est la somme du nombre de modalités d'attributs représenté par chaque couche}\\
		& U \text{ est l'emprunteur courant}.
	\end{align*}

Le tableau \ref{fig:gpaerup} ci-dessous présentent les comparaisons sur cinq jeux de données (AER, AFB, CREDIT RISK, GERMAN, JAPAN), de l'impact de l'usage des descripteurs issus des graphes multicouches lorsque le PageRank est appliqué de manière globale pour tous les noeuds et lorsque qu'il est appliqué de manière personnalisée pour chaque prêt. 

Suivant la métrique $Accuracy$, l'intégration des descripteurs issus du graphe multicouches lorsque le PageRank est appliqué de manière globale permet de faire mieux que le cas classique (uniquement avec les descripteurs du jeu de données) dans 18 cas sur 50 (36\%), et le PageRank personnalisé par prêt dans 22 cas sur 50 (44\%). Concernant les autres métriques, on a respectivement 14/50 (28\%) et 25/50 (50\%) suivant la métrique $Précision$, et enfin 15/50 (30\%) et 23/50 (46\%) suivant les métriques $Rappel$ et $F1-score$ pour le PageRank global pour tous les noeuds et le PageRank personnalisé par prêt.

Ces résultats montrent que l'usage des descripteurs extraits du graphe multicouches permet d'améliorer les performances des modèles classiques de prédiction du risque de crédit au moins dans 28\% des cas si on utilise le PageRank globale pour tous les noeuds et dans au moins 44\% des cas si on utilise le PageRank personnalisé par prêt. Ceci confirme l'intérêt de procéder à la personnalisation du PageRank par prêt.

Les pourcentages d'amélioration des techniques classiques de prédiction du risque de crédit par l'usage des descripteurs extraits des graphes multicouches peuvent atteindre jusqu'à 0.3\% dans le jeu de données $AFB$ de Afriland First Bank, 1\% dans le jeu de données $CREDIT\ RISK$, 4\% dans le jeu de données $AER$, 6.5\% dans le jeu de données $JAPAN$ et enfin 23.4\% dans le jeu de données $GERMAN$.

\newpage
\thispagestyle{empty}
\begin{table}[H]
  \centering
  \includegraphics[width=1.03\textwidth]{assets/PageRank_G_vs_P}
  \caption{Comparaison des pourcentage d'amélioration de la performance du cas classique (Classic) par les graphes multicouches en utilisant le PageRank global pour tous les noeuds (G) et le PageRank personnalisé pour chaque prêt (P), dans les 05 jeux de données considérés.}
  \label{fig:gpaerup}
\end{table}

\newpage
Après analyse du tableau général des résultats obtenus pour évaluer l'apport de l'intégration des descripteurs extraits des graphes multicouches, nous nous intéressons à la contribution de ces nouveaux descripteurs dans les processus de décision des techniques classiques de prédiction du risque de crédit bancaire. Pour ce faire, nous construisons les diagrammes SHAP des valeurs de Shapley de tous les descripteurs positionnés en entrée des modèles de prédiction. 

Les figures \ref{fig:shap1}, \ref{fig:shap2}, \ref{fig:shap3}, \ref{fig:shap4} et \ref{fig:shap5} présentent les diagrammes SHAP des Top-20 des valeurs de shapley des descripteurs des modèles de prédiction de type \textbf{Classic + MLN} respectivement basé sur $SVM$, \textit{Régression Logistique}, \textit{Arbre de Décision}, \textit{Fôret d'Arbres de Décision} et $XGBoost$. L'analyse de ces figures permet de faire les remarques et observations suivantes :

\begin{itemize}
	\item AER : Nous remarquons de part ses graphiques dans le contexte du jeu de données AER, les descripteurs issues des graphes ont une tendance à ne pas forte expliquer autant les décision de possibilité défaut de prêt ou de solvabilité. De plus la meilleur explication est obtenu à partir du modèle \ac{SVM}
	\item AFB : Pour ce qui concerne AFB, les descripteurs issus des graphes font partir des attributs qui expliquent le mieux la décision des modèles. Encore plus loin, nous remarquons que les descripteurs résultant du PageRank personnalisé à un seul emprunt surpasse la version de l'article de base.
	\item CREDIT RISK : Ces résultats présentent les mêmes caractéristiques que la AFB, et une fois de plus, nous avons la supériorité des descripteurs du PageRank avec personnalisation à un emprunt.
	\item GERMAN : Les modèles basé sur des arbres pour ce jeux de données sont ceux pour qui les attributs des graphes expliquent le mieux les sortir en particulier la version issus de la personnalisation à un seul emprunt du PageRank
	\item JAPAN : Les attributs des graphes de façon généraliste permettent d'expliquer les sorties pour ce jeu de données. Et une fois de plus, PageRank personnalisé à un seul emprunt surpasse la version à personnalisation globale de l'article.
\end{itemize}

\newpage
\begin{figure}[H]
  \label{fig:shap1}
  \centering
  \includegraphics[width=1.06\textwidth]{assets/sv_shapley.png}
  \caption{Diagramme SHAP de l'importance des attributs pour le modèle SVM}
\end{figure}

\begin{figure}[H]
  \label{fig:shap2}
  \centering
  \includegraphics[width=1.06\textwidth]{assets/lrc_shapley.png}
  \caption{Diagramme SHAP de l'importance des attributs pour la Régression Logistique}
\end{figure}

\begin{figure}[H]
  \label{fig:shap3}
  \centering
  \includegraphics[width=1.06\textwidth]{assets/dtc_shapley.png}
  \caption{Diagramme SHAP de l'importance des attributs pour l'Arbre de Décision}
\end{figure}

\begin{figure}[H]
  \label{fig:shap4}
  \centering
  \includegraphics[width=1.06\textwidth]{assets/rfc_shapley.png}
  \caption{Diagramme SHAP de l'importance des attributs pour la Fôret Aléatoire}
\end{figure}

\begin{figure}[H]
  \label{fig:shap5}
  \centering
  \includegraphics[width=1.06\textwidth]{assets/xgb_shapley.png}
  \caption{Diagramme SHAP de l'importance des attributs pour XGBoost}
\end{figure}

En somme, nous remarquons de part ses graphiques que les descripteurs issus des graphes en moyenne pour chaque jeu de données se trouve dans le top 10. Cependant nous devons mettre une emphase sur les descripteurs ultra personnalisé qui eux en générale se trouve dans le top 5 pour chaque modèle d'apprentissage que se soit en faveur de la classe positive ou négative. Ceci dis que pourrait apporter le choix d'une meilleur configuration du \ac{MLN} en terme de pertinence de descripteurs extraits et de métriques de classification?

\subsection{Choix du nombre de couche}
\label{sec:cc}
Pour expérimenter l'impact du nombreux de couches sur la qualité des descripteurs choisis, nous avons construire un graphe multicouches à 1 seule couche ou un seul attribut catégorielle (\ac{MLN} 1) et un autre graphe multicouches avec autant de couche qu'il y'a d'attributs de type dans l'ensemble de données (\ac{MLN} All).
\subsubsection{\ac{MLN} 1}
Comme mise en hypothèse, il peut s'avérer que les relations entre les emprunts soient essentiellement porter sur un seul niveau, caractéristique ou couche. Alors, pour verifier cela, nous avons modélisé nos données sous forme de graphes multicouches biparti à une seule couche. 

Pour chaque attribut catégoriel du jeu de données, nous construisons ces graphes et extrayons les informations suivante où \textbf{case\_k} représente le nom de l'attribut qui sert à construire le \ac{MLN} 1:
\begin{itemize}
		\item \textbf{MLN\_case\_k\_degré :} le nombre d'emprunteur avec la même case\_k
		\item \textbf{MLN\_bipart\_intra\_case\_k :} le score PageRank du noeud de l'emprunt de couche case\_k lorsque seul les noeuds intra (modalités motif) sont inclus dans le vecteur de personnalisation du PageRank
		\item \textbf{MLN\_bipart\_inter\_case\_k :} le score PageRank du noeud de l'emprunt de couche case\_k lorsque seul les noeuds intra (noeuds emprunt) sont inclus dans le vecteur de personnalisation du PageRank
		\item \textbf{MLN\_bipart\_combine\_case\_k :} le score PageRank du noeud de l'emprunt de couche case\_k 
		\item \textbf{MLN\_bipart\_intra\_max\_case\_k :} le score PageRank du noeud de la modalité de couche case\_k lorsque seul les noeuds intra (modalités motif) sont inclus dans le vecteur de personnalisation du PageRank
		\item \textbf{MLN\_bipart\_inter\_max\_case\_k :} le score PageRank du noeud de la modalité de couche case\_k lorsque seul les noeuds intra (noeuds emprunt) sont inclus dans le vecteur de personnalisation du PageRank
		\item \textbf{MLN\_bipart\_combine\_max\_case\_k :} le score PageRank du noeud de la modalité de couche case\_k
		\item \textbf{MLN\_bipart\_ultra\_case\_k :} le score PageRank du noeud de l'emprunt de couche case\_k lorsque seul les noeuds associés à l'emprunt (un seul emprunteur) sont inclus dans le vecteur de personnalisation du PageRank
		\item \textbf{MLN\_bipart\_ultra\_max\_case\_k :} le score PageRank du noeud de la modalité de couche case\_k lorsque seul les noeuds associés à l'emprunt (un seul emprunteur) sont inclus dans le vecteur de personnalisation du PageRank
	\end{itemize}

\subsubsection{\ac{MLN} All }
Nous expérimentons dans cette partie, l'impact d'une modélisation des emprunts avec autant de couches qu'il en a de données catégorielles sur l'évaluation du risque de crédit financier. Chaque couche dans cette modélisation met en relation chaque emprunt à sa modalité de la caractéristique de la couche et les différente couche communiquent entre elles par le biais des noeuds matérialisant l'emprunt. Comme dans \ac{MLN} 1, nous avons extraire les même descripteurs.
\subsubsection{Résultats}

\begin{table}[H]
  \centering
  \includegraphics[width=\textwidth]{assets/AER_MLN_UP}
  \caption{Gain d'impacte des modélisations pour le jeu de données AER de Kaggle}
  \label{fig:gimae}
\end{table}


\begin{table}[H]
  \centering
  \includegraphics[width=\textwidth]{assets/AFB_MLN_UP}
  \caption{Gain d'impacte des modélisations pour le jeu de données AFB}
  \label{fig:gimaf}
\end{table}


\begin{table}[H]
  \centering
  \includegraphics[width=\textwidth]{assets/CREDITR_MLN_UP}
  \caption{Gain d'impacte des modélisations pour le jeu de données CREDIT RISK DATASET}
  \label{fig:gimc}
\end{table}


\begin{table}[H]
  \centering
  \includegraphics[width=\textwidth]{assets/GERMAN_MLN_UP}
  \caption{Gain d'impacte des modélisations pour le jeu de données GERMAN}
  \label{fig:gimg}
\end{table}

\begin{table}[H]
  \centering
  \includegraphics[width=\textwidth]{assets/JAPAN_MLN_UP}
  \caption{Gain d'impacte des modélisations pour le jeu de données JAPAN}
  \label{fig:gimj}
\end{table}

Les figure ci-contre nous montre que dans certains cas, la modélisation à 1 couche peut permettre d'extraire des motifs du graphe qui améliores les métriques de classification. Cependant il apparait clairement que le fait de modéliser une relation des emprunts sur la base de toutes les attributs catégoriels possible n'est pas vraiment une solution car n'améliore pas vraiment les modèles classiques. Il serait donc judicieux de pouvoir identifier quel sont les attributs catégoriels qui garantissent d'apporter de meilleurs résultats

\subsection{Choix des attributs à considérer}
Le choix optimal des attributs devant servir à modéliser nos données de prêts peut vraiment avoir de l'impact sur les modèle prédictif résultant et même la qualité des motifs extraits des graphes construits. Pour acter ce besoin, nous allons sur l'intuition de la disponibilité de correlation entre l'impact de la modélisation d'un attribut en \ac{MLN} 1 et la qualité des couplage d'ordre $k_{>=2}$. Dans un sens, nous avons proposé un clustering des impacts de modélisation à une couche pour voire observer la co-occurence des comportements des couplages à la couche $ k \gets 2$ dans le cas de nos expérimentations. 

%\begin{figure}[H]
 % \centering
 % \includegraphics[width=\textwidth]{assets/AER_C}
 % \caption{Correlation entre l'impact d'attributs en \ac{MLN} 1 et celui de des couples en \ac{MLN} 2  pour le jeu de données AER de Kaggle}
 % \label{fig:cae}
%\end{figure}


%\begin{figure}[H]
%  \centering
%  \includegraphics[width=\textwidth]{assets/AFB_C}
%  \caption{Correlation entre l'impact d'attributs en \ac{MLN} 1 et celui de des couples en \ac{MLN} 2  pour le jeu de données AFB}
%  \label{fig:caf}
%\end{figure}


%\begin{figure}[H]
%  \centering
%  \includegraphics[width=\textwidth]{assets/CREDITR_C}
%  \caption{Correlation entre l'impact d'attributs en \ac{MLN} 1 et celui de des couples en \ac{MLN} 2  pour le jeu de données CREDIT RISK DATASET}
  %\label{fig:cc}
%\end{figure}


%\begin{figure}[H]
%  \centering
%  \includegraphics[width=\textwidth]{assets/GERMAN_C}
%  \caption{Correlation entre l'impact d'attributs en \ac{MLN} 1 et celui de des couples en \ac{MLN} 2  pour le jeu de données GERMAN}
%  \label{fig:cg}
%\end{figure}

%\begin{figure}[H]
%  \centering
%  \includegraphics[width=\textwidth]{assets/JAPAN_C}
%  \caption{Correlation entre l'impact d'attributs en \ac{MLN} 1 et celui de des couples en \ac{MLN} 2  pour le jeu de données JAPAN}
%  \label{fig:cj}
%\end{figure}

\begin{table}[H]
  \centering
  \includegraphics[width=\textwidth]{assets/ALL_C}
  \caption{Correlation entre l'impact d'attributs en \ac{MLN} 1 et celui de des couples en \ac{MLN} 2  pour tous les jeux de données}
  \label{fig:ca}
\end{table}

Nous remarquons suivant la figure~\ref{fig:ca}, nous remarquons que les couples formés d'attributs ayant un bon impact en \ac{MLN} 1 (Good \ac{MLN} 1, Good \ac{MLN} 1) ont tendances à améliorer les modèles classiques en \ac{MLN} 2.


\subsection{Les meilleurs performances}
Des resultats des sections~\ref{sec:pp},~\ref{sec:cc}, nous pouvons extraire les tendances suivantes:

\begin{figure}[H]
  \centering
  \includegraphics[width=\textwidth]{assets/best_results}
   \captionsetup{labelformat=figure}
  \caption{Importance des modèles, logiques, approches, couches dans l'évaluation du risque de crédit.}
  \label{fig:bt}
\end{figure}

Premièrement le modèle \ac{XGB} est le modèle qui généralise au mieux l'ensemble des données des différent jeux de données. Ensuite \textbf{Classic + MLN} permet d'optimiser les métriques sur les différents jeux de données. Puis nous remarquons que \textbf{la personnalisation à un seul emprunteur} surpasse la personnalisation global sur nos jeux de données. Enfin la modélisation d'un \textbf{\ac{MLN} 2} permet de mieux capturer des motifs dans les graphes issus des données de prêts des différents jeux de données.

% Conclusion et perspectives
\section{Conclusion et perspectives}
\label{sec:con}
Au terme de cette rédaction, il était question pour nous de présenter les différentes expérimentation réalisées au cours de ses derniers mois et leur résultats. Il en ressort que l'idée de personnalisation à un seul emprunteur apparaît comme un intuition valide mais l'apport n'est pas très éloigné de la version global proposé par les auteurs de l'article de base mais s'adapte mieux sur tout jeu de données. La modélisation à une couche nous aura permis de comprendre que si nous voulons construire des \ac{MLN} $k_{>=2}$, il serait judicieux de prendre ceux là donc la version \ac{MLN} 1 améliore les modèles classiques.

Lors de ces expérimentation, Nous avons remarqué la non importance de plusieurs attributs des jeux de données dans les modèles, l'absence d'information de classe dans le graphe multicouches biparti construit, la minorité d'attributs extraits des graphes. Pour cela nous pensons au perspectives suivantes:
\begin{enumerate}
\item définir un processus de présélection des attributs avant entrainement
\item intégrer les classes dans la modélisation du graphe multicouches
\item exploitation différente du graphe multicouches sans avoir recourt au PageRank.
\end{enumerate}


% reférence
\newpage
\bibliographystyle{IEEEtran}
\bibliography{sample}
\nocite{*}



	
\end{document}
