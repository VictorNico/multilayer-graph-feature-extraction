\documentclass[11pt]{beamer}
\usepackage[utf8]{inputenc}
\usepackage[T1]{fontenc}
\usepackage{lmodern}
\usepackage{translator}
\usepackage[french]{babel}
\usetheme{madrid}
\usepackage{fontspec}
\usepackage{color}
\usepackage{xcolor}
\setsansfont{Times New Roman}
\usepackage{graphics}
\usepackage{graphicx}
\usepackage{float}
\usepackage{booktabs}
\usepackage{subfigure}
\usepackage{multicol}
\usepackage{fancyhdr}
\usepackage{lipsum}
\usepackage{tikz}
\usepackage{wallpaper}
\usepackage{wrapfig}
\usepackage{multicol}
\usetikzlibrary{positioning}
\usepackage{algorithm}
\usepackage{algpseudocode}
\usepackage{acronym}

\begin{document}
	\author[V.DJIEMBOU]{DJIEMBOU TIENTCHEU VICTOR NICO\inst{1} }
	\title{Extraction de descripteurs des graphes}
	\subtitle{Cas de graphes Multicouches biparti}
	\titlegraphic{\includegraphics[scale=0.3]{assets/logo}}
	\institute[UYI]{\inst{1} \textit{Université de Yaoundé I,\\ Faculté des Sciences,\\ Département d'Informatique,\\ Étudiant Master SD} }
	%\date{}
	%\subject{}
	%\setbeamercovered{transparent}
	\setbeamertemplate{navigation symbols}{}
	\setbeamertemplate{itemize }{circle}
	\newcommand\Background{
            \begin{tikzpicture}[remember picture,overlay]
			\node[inner sep=0pt,outer sep=0pt,opacity=1]
			at (5.7,-4)
			{\includegraphics[scale=0.23]{assets/999}};
		\end{tikzpicture}
        }
	%\setbeamertemplate{background canvas}{\includegraphics[scale=0.19]{999}}
	\setbeamercolor{frametitle}{bg=gray}
	%\AtBeginSection[]{
  	%	\begin{frame}
    	%		\frametitle{Outline}
   	%		 \tableofcontents[currentsection]
  	%	\end{frame}
	%}
	
	% Définition des acronymes
\acrodef{MLN}{MultiLayer Network}
	
\begin{frame}[plain]
	\Background
	\maketitle
\end{frame}

\begin{frame}
	\Background
	\tableofcontents
\end{frame}
% Introduction

\section{Graphes Multi-couches Biparti}
\subsection{Définition}
\begin{frame}
	\frametitle{\textsc{Graphes Multicouches Biparti}}
	\begin{block}{Définition}
		\begin{itemize}
			\item \textbf{Noeud} : L'identificateur d'un emprunteur ou encore une modalité d'un attribut descriptif
			\item \textbf{Arête} : Matérialise qu'il existe une relation entre deux noeuds
			\item \textbf{Niveau ou Couche} : Ensemble d'emprunteur connecté à une modalité d'un attribut descriptif
		\end{itemize}
	\end{block}
	\begin{columns}[T]
    		\begin{column}{0.5\textwidth}
     			 \includegraphics[width=0.9\textwidth]{assets/bimln}
    		\end{column}
		\begin{column}{0.5\textwidth}
     			 \begin{exampleblock}{Descripteurs}
				\begin{itemize}
					\item \textbf{Centralité de degrée}
					\item \textbf{Centralité de proximité} 
					\item \textbf{Centralité d'intermédiarité}
					\item \textbf{Centralité de vecteur propre}
					\item \textbf{PageRank}
				\end{itemize}
			\end{exampleblock}
    		\end{column}
 	 \end{columns}
	
\end{frame}

\subsection{Descripteurs}
\begin{frame}
	\frametitle{\textsc{Descripteurs}}
	\begin{block}{Centralité de dégrée}
		Ensemble des utilisateurs ayant les mêmes caractéristiques.
	\end{block}
	\begin{block}{PageRank Personnalisé}
		\begin{equation}
    			PR_{i+1} = \frac{{1-d}}{N} \times P + d \times PR_{i} \times H 
		\end{equation}

		où :
		\begin{align*}
    			& PR_{i} \text{ est le PageRank à l'itération i} , \\
			& PR_{i+1} \text{ est le PageRank à l'itération i+1} , \\
    			& d \text{ est le facteur d'amortissement (typiquement } d = 0,85), \\
    			& N \text{ est le nombre total de pages dans le réseau} \\
    			& H \text{ est la matrice de transition du réseau }.\\
    			& P \text{ est le vecteur de probabilités de démarrage}.
		\end{align*}
	\end{block}	
\end{frame}

\begin{frame}
	\frametitle{\textsc{Descripteurs}}
	
	
\end{frame}

\subsection{Processus d'extraction des descripteurs}
\begin{frame}
	\frametitle{\textsc{Extraction du descripteur du PageRank personnalisé}}
	\small
	\begin{block}{Algorithme PageRank personnalisé avec l'approche itérative}
		\begin{algorithmic}[1]
			\Procedure{PageRank}{d,P}
				\State Initialise $PR$ avec des valeurs égales pour chaque noeud
				\Repeat
					\State $PR_{new} = \frac{{1-d}}{N} + d \times PR \times H \times P$
					\State Mettre à jour les valeurs de $PR$ : $PR \gets PR_{new}$
				\Until{convergence}
				\State \Return $PR$
			\EndProcedure
		\end{algorithmic}
	\end{block}
	\begin{exampleblock}{Critères de convergence}
		\begin{itemize}
			\item \textbf{Stagnation des solutions (epsillon= 0.0001)}
			\item \textbf{Nombre d'itération} 
			\item \textbf{...}
		\end{itemize}
	\end{exampleblock}
\end{frame}

\begin{frame}
	\frametitle{\textsc{Extraction du descripteur de Degrée}}
	\small
	\begin{block}{Algorithme d'extraction de la centralité de degrée}
		\begin{algorithmic}[1]
			\Procedure{Degree}{}
				\State Initialise $DG$ de taille égale au nombre de noeud emprunteurs
				\For{chaque noeud $i$}
					\State $DG[i] \gets $ nombre d'emprunteurs partageant les mêmes informations
				\EndFor
				\State \Return $DG$
			\EndProcedure
		\end{algorithmic}
	\end{block}
\end{frame}

\section{Application processus de l'extraction de descripteurs des graphes}
\subsection{Données - Pré-traitement}
\begin{frame}
	\frametitle{\textsc{Données - Pré-traitement}}
	\begin{block}{Données Traditionnel}
		  % Requires the booktabs if the memoir class is not being used
		  \begin{table}[htbp]
		     \centering
		     \tiny
		     \setlength{\heavyrulewidth}{0.2pt} % Adjust the line thickness of \toprule
		     %\topcaption{Table captions are better up top} % requires the topcapt package
		     \resizebox{\textwidth}{!}{%
		     \begin{tabular}{@{} lccr @{}} % Column formatting, @{} suppresses leading/trailing space
		        \toprule
		        %\multicolumn{4}{c}{Attributs des transactions} \\
		        %\cmidrule(lr){1-2} % Partial rule. (r) trims the line a little bit on the right; (l) & (lr) also possible
		        ID & person\_home\_ownership    & loan\_intent & loan\_status\\
		        \midrule
		        0 & RENT & PERSONAL & 1 \\
		        1 & OWN & EDUCATION & 0 \\
		        2 & MORTGAGE & MEDICAL & 1 \\
		        3 & RENT & MEDICAL & 1 \\
		        4 & OWN & VENTURE & 1 \\
		        5 & MORTGAGE & EDUCATION & 0 \\
		        6 & RENT & MEDICAL & 0\\
		        7 & RENT & PERSONAL & 1\\
		        8 & OWN & HOMEIMPROVEMENT & 1\\
		        \bottomrule
		     \end{tabular}}
		     \caption{Un extrait du jeu de données Crédit Risk du benchmark}
		     \label{tab:data}
		  \end{table}
	\end{block}
	
\end{frame}

\begin{frame}
	\frametitle{\textsc{Données - Pré-traitement}}
	\begin{block}{Données Binarisé}
		  % Requires the booktabs if the memoir class is not being used
		  \begin{table}[htbp]
		     \centering
		     \tiny
		     \setlength{\heavyrulewidth}{0.2pt} % Adjust the line thickness of \toprule
		     \resizebox{\textwidth}{!}{%
		     \begin{tabular}{@{} lccccccccr @{}} % Column formatting, @{} suppresses leading/trailing space
		        \toprule
		        ID & RENT & OWN & MORTGAGE & PERSONAL & EDUCATION & MEDICAL & VENTURE &  HOMEIMPROVEMENT & loan\_status\\
		        \midrule
		        0 & 1 & 0 & 0 & 1 & 0 & 0 & 0 & 0 & 1 \\ 
		        1 & 0 & 1 & 0 & 0 & 1 & 0 & 0 & 0 & 0 \\ 
		        2 & 0 & 0 & 1 & 0 & 0 & 1 & 0 & 0 & 1 \\
		        3 & 1 & 0 & 0 & 0 & 0 & 1 & 0 & 0 & 1 \\ 
		        4 & 0 & 1 & 0 & 0 & 0 & 0 & 1 & 0 & 1 \\ 
		        5 & 1 & 0 & 1 & 0 & 1 & 0 & 0 & 0 & 0 \\ 
		        6 & 1 & 0 & 0 & 0 & 0 & 1 & 0 & 0 & 0 \\ 
		        7 & 1 & 0 & 0 & 1 & 0 & 0 & 0 & 0 & 1 \\ 
		        8 & 0 & 1 & 0 & 0 & 0 & 0 & 0 & 1 & 1 \\  
		        \bottomrule
		     \end{tabular}}
		     \caption{Version OHE de l'extrait du jeu de données Crédit Risk du benchmark}
		     \label{tab:ohe}
		  \end{table}
	\end{block}
	
\end{frame}

\subsection{Construction d'un graphe multi-couche (\ac{MLN}) biparti à k couches }
\begin{frame}
	\frametitle{\textsc{Construction d'un graphe multi-couches biparti à k couches}}
	\begin{block}{\ac{MLN} à 1 couche de person\_home\_ownership}
	\resizebox{\textwidth}{!}{%
	\begin{tikzpicture}[
  		node distance=1cm, % Espacement entre les nœuds
  		every node/.style={draw, circle, minimum size=0.3cm, fill=red!30}, % Style des nœuds emprunteurs
		modality/.style={draw, circle, minimum size=0.3cm, distance=1cm, fill=blue!30}, % Style des nœuds modalité couche 1
  		edge/.style={shorten >=2pt, shorten <=2pt} % Style des arêtes avec une longueur de 2 cm
	]
  
  		% Noeuds emprunteurs
    		\node (C0U1) {C0U1};
		\node[right=of C0U1] (C0U2) {C0U2};
		\node[right=of C0U2] (C0U3) {C0U3};
  		\node[right=of C0U3] (C0U4) {C0U4};
		\node[right=of C0U4] (C0U5) {C0U5};
		\node[right=of C0U5] (C0U6) {C0U6};
		\node[right=of C0U6] (C0U7) {C0U7};
		\node[right=of C0U7] (C0U8) {C0U8};
		\node[right=of C0U8] (C0U9) {C0U9};
		
		
  		% Noeuds de modalité
  		\node[below=of C0U1,modality] (MORTGAGE) {MORTGAGE};
		\node[right=of MORTGAGE, modality] (OWN) {OWN};
		\node[right=of OWN, modality] (RENT)  {RENT};
		
		
  		% Arêtes
  		\draw (C0U1) -- (RENT);
		\draw (C0U2) -- (OWN);
		\draw (C0U3) -- (MORTGAGE);
		\draw (C0U4) -- (RENT);
		\draw (C0U5) -- (OWN);
		\draw (C0U6) -- (MORTGAGE);
		\draw (C0U7) -- (RENT);
		\draw (C0U8) -- (RENT);
		\draw (C0U9) -- (OWN);
		
	\end{tikzpicture}}
	\end{block}
	
\end{frame}
\begin{frame}
	\frametitle{\textsc{Construction d'un graphe multi-couches à k couches}}
	\begin{block}{MLN à 1 couche de loan\_intent}
	\resizebox{\textwidth}{!}{%
	\begin{tikzpicture}[
  		node distance=1cm, % Espacement entre les nœuds
  		every node/.style={draw, circle, minimum size=0.3cm, fill=red!30}, % Style des nœuds emprunteurs
		modality/.style={draw, circle, minimum size=0.3cm, distance=1cm, fill=green!30}, % Style des nœuds modalité couche
  		edge/.style={shorten >=2pt, shorten <=2pt} % Style des arêtes avec une longueur de 2 cm
	]
  
  		% Noeuds emprunteurs
    		\node (C1U1) {C1U1};
		\node[right=of C1U1] (C1U2) {C1U2};
		\node[right=of C1U2] (C1U3) {C1U3};
  		\node[right=of C1U3] (C1U4) {C1U4};
		\node[right=of C1U4] (C1U5) {C1U5};
		\node[right=of C1U5] (C1U6) {C1U6};
		\node[right=of C1U6] (C1U7) {C1U7};
		\node[right=of C1U7] (C1U8) {C1U8};
		\node[right=of C1U8] (C1U9) {C1U9};
		
		
  		% Noeuds de modalité
  		\node[below=of C1U1,modality] (PERSONAL) {PERSONAL};
		\node[right=of PERSONAL, modality] (EDUCATION) {EDUCATION};
		\node[right=of EDUCATION, modality] (MEDICAL) {MEDICAL};
		\node[right=of MEDICAL, modality] (VENTURE) {VENTURE};
		\node[right=of VENTURE, modality] (HOMEIMPROVEMENT)  {HOMEIMPROVEMENT};
		
		
  		% Arêtes
  		\draw (C1U1) -- (PERSONAL);
		\draw (C1U2) -- (EDUCATION);
		\draw (C1U3) -- (MEDICAL);
		\draw (C1U4) -- (MEDICAL);
		\draw (C1U5) -- (VENTURE);
		\draw (C1U6) -- (EDUCATION);
		\draw (C1U7) -- (MEDICAL);
		\draw (C1U8) -- (PERSONAL);
		\draw (C1U9) -- (HOMEIMPROVEMENT);
		
	\end{tikzpicture}}
	\end{block}
	
\end{frame}
\begin{frame}
	\frametitle{\textsc{Construction d'un graphe multi-couches à k couches}}
	\begin{block}{MLN à 2 couches de loan\_intent et de person\_home\_ownership}
	\resizebox{\textwidth}{!}{%
	\begin{tikzpicture}[
  		node distance=1cm, % Espacement entre les nœuds
  		every node/.style={draw, circle, minimum size=0.3cm, fill=red!30}, % Style des nœuds emprunteurs
		modality1/.style={draw, circle, minimum size=0.3cm, distance=1cm, fill=blue!30}, % Style des nœuds modalité couche 1
		modality2/.style={draw, circle, minimum size=0.3cm, distance=1cm, fill=green!30}, % Style des nœuds modalité couche 2
  		edge/.style={shorten >=2pt, shorten <=2pt} % Style des arêtes avec une longueur de 2 cm
	]
  		% couche 1
  		% Noeuds emprunteurs
    		\node (C1U1) {C1U1};
		\node[right=of C1U1] (C1U2) {C1U2};
		\node[right=of C1U2] (C1U3) {C1U3};
  		\node[right=of C1U3] (C1U4) {C1U4};
		\node[right=of C1U4] (C1U5) {C1U5};
		\node[right=of C1U5] (C1U6) {C1U6};
		\node[right=of C1U6] (C1U7) {C1U7};
		\node[right=of C1U7] (C1U8) {C1U8};
		\node[right=of C1U8] (C1U9) {C1U9};
		
		
  		% Noeuds de modalité
  		\node[above=of C1U1,modality2] (PERSONAL) {PERSONAL};
		\node[right=of PERSONAL, modality2] (EDUCATION) {EDUCATION};
		\node[right=of EDUCATION, modality2] (MEDICAL) {MEDICAL};
		\node[right=of MEDICAL, modality2] (VENTURE) {VENTURE};
		\node[right=of VENTURE, modality2] (HOMEIMPROVEMENT)  {HOMEIMPROVEMENT};
		
		
  		% Arêtes
  		\draw (C1U1) -- (PERSONAL);
		\draw (C1U2) -- (EDUCATION);
		\draw (C1U3) -- (MEDICAL);
		\draw (C1U4) -- (MEDICAL);
		\draw (C1U5) -- (VENTURE);
		\draw (C1U6) -- (EDUCATION);
		\draw (C1U7) -- (MEDICAL);
		\draw (C1U8) -- (PERSONAL);
		\draw (C1U9) -- (HOMEIMPROVEMENT);
		
		% couche 2
		% Noeuds emprunteurs
    		\node[below=of C1U1] (C0U1) {C0U1};
		\node[right=of C0U1] (C0U2) {C0U2};
		\node[right=of C0U2] (C0U3) {C0U3};
  		\node[right=of C0U3] (C0U4) {C0U4};
		\node[right=of C0U4] (C0U5) {C0U5};
		\node[right=of C0U5] (C0U6) {C0U6};
		\node[right=of C0U6] (C0U7) {C0U7};
		\node[right=of C0U7] (C0U8) {C0U8};
		\node[right=of C0U8] (C0U9) {C0U9};
		
		
  		% Noeuds de modalité
  		\node[below=of C0U1,modality1] (MORTGAGE) {MORTGAGE};
		\node[right=of MORTGAGE, modality1] (OWN) {OWN};
		\node[right=of OWN, modality1] (RENT)  {RENT};
		
		
  		% Arêtes
  		\draw (C0U1) -- (RENT);
		\draw (C0U2) -- (OWN);
		\draw (C0U3) -- (MORTGAGE);
		\draw (C0U4) -- (RENT);
		\draw (C0U5) -- (OWN);
		\draw (C0U6) -- (MORTGAGE);
		\draw (C0U7) -- (RENT);
		\draw (C0U8) -- (RENT);
		\draw (C0U9) -- (OWN);
		
		% Arêtes intercouche
		% Arêtes
  		\draw (C0U1) -- (C1U1);
		\draw (C0U2) -- (C1U2);
		\draw (C0U3) -- (C1U3);
		\draw (C0U4) -- (C1U4);
		\draw (C0U5) -- (C1U5);
		\draw (C0U6) -- (C1U6);
		\draw (C0U7) -- (C1U7);
		\draw (C0U8) -- (C1U8);
		\draw (C0U9) -- (C1U9);
		
	\end{tikzpicture}}
	\end{block}
	
\end{frame}

\subsection{Extraction de descripteurs - Augmentation de données}
\begin{frame}
	\frametitle{\textsc{Extraction de descripteurs du descripteur Degrée}}
	%\resizebox{\textwidth}{!}{%
	%\begin{block}{Degrée de U1}
	\small
	\begin{equation}
   		Degree_{U_i} =  \lvert \{U_i :  \exists (U_i, k_i)\} \rvert - 1
	\end{equation}
	tel que $k_i$ soit une des modalités associées à l'emprunteur  $U_i$ dans les $k$ couches 
	%\end{block}
	\resizebox{\textwidth}{!}{%
	\begin{tikzpicture}[
  		node distance=1cm, % Espacement entre les nœuds
  		every node/.style={draw, circle, minimum size=0.3cm, fill=red!30}, % Style des nœuds emprunteurs
		modality1/.style={draw, circle, minimum size=0.3cm, distance=1cm, fill=blue!30}, % Style des nœuds modalité couche 1
		modality2/.style={draw, circle, minimum size=0.3cm, distance=1cm, fill=green!30}, % Style des nœuds modalité couche 2
  		edge/.style={shorten >=2pt, shorten <=2pt} % Style des arêtes avec une longueur de 2 cm
	]
  		% couche 1
  		% Noeuds emprunteurs
    		\node (C1U1) {C1U1};
		\node[right=of C1U1] (C1U2) {C1U2};
		\node[right=of C1U2] (C1U3) {C1U3};
  		\node[right=of C1U3] (C1U4) {C1U4};
		\node[right=of C1U4] (C1U5) {C1U5};
		\node[right=of C1U5] (C1U6) {C1U6};
		\node[right=of C1U6] (C1U7) {C1U7};
		\node[right=of C1U7] (C1U8) {C1U8};
		\node[right=of C1U8] (C1U9) {C1U9};
		
		
  		% Noeuds de modalité
  		\node[above=of C1U1,modality2] (PERSONAL) {PERSONAL};
		\node[right=of PERSONAL, modality2] (EDUCATION) {EDUCATION};
		\node[right=of EDUCATION, modality2] (MEDICAL) {MEDICAL};
		\node[right=of MEDICAL, modality2] (VENTURE) {VENTURE};
		\node[right=of VENTURE, modality2] (HOMEIMPROVEMENT)  {HOMEIMPROVEMENT};
		
		
  		% Arêtes
  		\draw[draw=red, thick] (C1U1) -- (PERSONAL);
		\draw (C1U2) -- (EDUCATION);
		\draw (C1U3) -- (MEDICAL);
		\draw (C1U4) -- (MEDICAL);
		\draw (C1U5) -- (VENTURE);
		\draw (C1U6) -- (EDUCATION);
		\draw (C1U7) -- (MEDICAL);
		\draw[draw=red, thick] (C1U8) -- (PERSONAL);
		\draw (C1U9) -- (HOMEIMPROVEMENT);
		
		% couche 2
		% Noeuds emprunteurs
    		\node[below=of C1U1] (C0U1) {C0U1};
		\node[right=of C0U1] (C0U2) {C0U2};
		\node[right=of C0U2] (C0U3) {C0U3};
  		\node[right=of C0U3] (C0U4) {C0U4};
		\node[right=of C0U4] (C0U5) {C0U5};
		\node[right=of C0U5] (C0U6) {C0U6};
		\node[right=of C0U6] (C0U7) {C0U7};
		\node[right=of C0U7] (C0U8) {C0U8};
		\node[right=of C0U8] (C0U9) {C0U9};
		
		
  		% Noeuds de modalité
  		\node[below=of C0U1,modality1] (MORTGAGE) {MORTGAGE};
		\node[right=of MORTGAGE, modality1] (OWN) {OWN};
		\node[right=of OWN, modality1] (RENT)  {RENT};
		
		
  		% Arêtes
  		\draw[draw=red, thick] (C0U1) -- (RENT);
		\draw (C0U2) -- (OWN);
		\draw (C0U3) -- (MORTGAGE);
		\draw[draw=red, thick] (C0U4) -- (RENT);
		\draw (C0U5) -- (OWN);
		\draw (C0U6) -- (MORTGAGE);
		\draw[draw=red, thick] (C0U7) -- (RENT);
		\draw[draw=red, thick] (C0U8) -- (RENT);
		\draw (C0U9) -- (OWN);
		
		% Arêtes intercouche
		% Arêtes
  		\draw (C0U1) -- (C1U1);
		\draw (C0U2) -- (C1U2);
		\draw (C0U3) -- (C1U3);
		\draw (C0U4) -- (C1U4);
		\draw (C0U5) -- (C1U5);
		\draw (C0U6) -- (C1U6);
		\draw (C0U7) -- (C1U7);
		\draw (C0U8) -- (C1U8);
		\draw (C0U9) -- (C1U9);
		
	\end{tikzpicture}}
\end{frame}

\subsubsection{Extraction de descripteurs du descripteur Degrée}
\begin{frame}
	\frametitle{\textsc{Extraction de descripteurs du descripteur Degrée}}
	\begin{block}{Degrée de chaque emprunteur}
		\begin{itemize}
			\item $DG_{U1} = \lvert  \{U8, U1\} \cap \{U8, U1, U7, U4\} \rvert - 1 = 1$
			\item $DG_{U2} = \lvert  \{U6, U2\} \cap \{U5, U2, U9\} \rvert - 1 = 0$
			\item $DG_{U3} = \lvert  \{U7, U3, U4\} \cap \{U6, U3\} \rvert - 1 = 0$
			\item $DG_{U4} = \lvert  \{U7, U4, U3\} \cap \{U8, U1, U7, U4\} \rvert - 1 = 1$
			\item $DG_{U5} = \lvert  \{U5\} \cap \{U5, U2\} \rvert - 1 = 0$
			\item $DG_{U6} = \lvert  \{U2, U6\} \cap \{U6, U3\} \rvert - 1 = 0$
			\item $DG_{U7} = \lvert  \{U7, U3, U4\} \cap \{U8, U1, U7, U4\} \rvert - 1 = 1$
			\item $DG_{U8} = \lvert  \{U8, U1\} \cap \{U8, U1, U7, U4\} \rvert - 1 = 1$
			\item $DG_{U9} = \lvert  \{U9\} \cap \{U5, U2, U9\} \rvert - 1 = 0$
		\end{itemize}
		
		\[
			DG = 
			\begin{pmatrix} 1 & 0 & 0 & 1 & 0 & 0 & 1 & 1 & 0 \end{pmatrix}
		\]
	\end{block}
\end{frame}

\subsubsection{Extraction du descripteur PageRank personnalisé}
\begin{frame}
	\frametitle{\textsc{Extraction du descripteur PageRank personnalisé}}
	\small
	\resizebox{\textwidth}{!}{%
	\begin{tikzpicture}[
  		node distance=1cm, % Espacement entre les nœuds
  		every node/.style={draw, circle, minimum size=0.3cm, fill=red!30}, % Style des nœuds emprunteurs
		modality1/.style={draw, circle, minimum size=0.3cm, distance=1cm, fill=blue!30}, % Style des nœuds modalité couche 1
		modality2/.style={draw, circle, minimum size=0.3cm, distance=1cm, fill=green!30}, % Style des nœuds modalité couche 2
  		edge/.style={shorten >=2pt, shorten <=2pt} % Style des arêtes avec une longueur de 2 cm
	]
  		% couche 1
  		% Noeuds emprunteurs
    		\node (C1U1) {C1U1};
		\node[right=of C1U1] (C1U2) {C1U2};
		\node[right=of C1U2] (C1U3) {C1U3};
  		\node[right=of C1U3] (C1U4) {C1U4};
		\node[right=of C1U4] (C1U5) {C1U5};
		\node[right=of C1U5] (C1U6) {C1U6};
		\node[right=of C1U6] (C1U7) {C1U7};
		\node[right=of C1U7] (C1U8) {C1U8};
		\node[right=of C1U8] (C1U9) {C1U9};
		
		
  		% Noeuds de modalité
  		\node[above=of C1U1,modality2] (PERSONAL) {PERSONAL};
		\node[right=of PERSONAL, modality2] (EDUCATION) {EDUCATION};
		\node[right=of EDUCATION, modality2] (MEDICAL) {MEDICAL};
		\node[right=of MEDICAL, modality2] (VENTURE) {VENTURE};
		\node[right=of VENTURE, modality2] (HOMEIMPROVEMENT)  {HOMEIMPROVEMENT};
		
		
  		% Arêtes
  		\draw (C1U1) -- (PERSONAL);
		\draw (C1U2) -- (EDUCATION);
		\draw (C1U3) -- (MEDICAL);
		\draw (C1U4) -- (MEDICAL);
		\draw (C1U5) -- (VENTURE);
		\draw (C1U6) -- (EDUCATION);
		\draw (C1U7) -- (MEDICAL);
		\draw (C1U8) -- (PERSONAL);
		\draw (C1U9) -- (HOMEIMPROVEMENT);
		
		% couche 2
		% Noeuds emprunteurs
    		\node[below=of C1U1] (C0U1) {C0U1};
		\node[right=of C0U1] (C0U2) {C0U2};
		\node[right=of C0U2] (C0U3) {C0U3};
  		\node[right=of C0U3] (C0U4) {C0U4};
		\node[right=of C0U4] (C0U5) {C0U5};
		\node[right=of C0U5] (C0U6) {C0U6};
		\node[right=of C0U6] (C0U7) {C0U7};
		\node[right=of C0U7] (C0U8) {C0U8};
		\node[right=of C0U8] (C0U9) {C0U9};
		
		
  		% Noeuds de modalité
  		\node[below=of C0U1,modality1] (MORTGAGE) {MORTGAGE};
		\node[right=of MORTGAGE, modality1] (OWN) {OWN};
		\node[right=of OWN, modality1] (RENT)  {RENT};
		
		
  		% Arêtes
  		\draw (C0U1) -- (RENT);
		\draw (C0U2) -- (OWN);
		\draw (C0U3) -- (MORTGAGE);
		\draw (C0U4) -- (RENT);
		\draw (C0U5) -- (OWN);
		\draw (C0U6) -- (MORTGAGE);
		\draw (C0U7) -- (RENT);
		\draw (C0U8) -- (RENT);
		\draw (C0U9) -- (OWN);
		
		% Arêtes intercouche
		% Arêtes
  		\draw (C0U1) -- (C1U1);
		\draw (C0U2) -- (C1U2);
		\draw (C0U3) -- (C1U3);
		\draw (C0U4) -- (C1U4);
		\draw (C0U5) -- (C1U5);
		\draw (C0U6) -- (C1U6);
		\draw (C0U7) -- (C1U7);
		\draw (C0U8) -- (C1U8);
		\draw (C0U9) -- (C1U9);
		
		\draw (-2, -7) rectangle (8, 4);
		
	\end{tikzpicture}}
\end{frame}

\begin{frame}
	\frametitle{\textsc{Extraction du descripteur PageRank personnalisé}}
	\begin{block}{Variables et Paramètres}
		\tiny
		\[
			P_{U1} = 
			\begin{bmatrix} 
			PERSONAL:1 \\
			EDUCATION:0 \\ 
			MEDICAL:0 \\
			C1U1: 1\\
			C1U2:0\\ 
			C1U3:0\\ 
			C1U4:0\\ 
			MORTGAGE:0 \\
			OWN:0\\
			RENT:1\\
			C0U1: 1\\
			C0U2:0\\ 
			C0U3:0\\ 
			C0U4:0\\ 
			\end{bmatrix} 
			=
			\begin{bmatrix} 
			1 \\
			0 \\ 
			0 \\
			1\\
			0\\ 
			0\\ 
			0\\ 
			0\\ 
			0\\
			1\\
			1\\
			0\\ 
			0\\ 
			0 
			\end{bmatrix} 
			;
			PR_0 = \frac{1}{14} \times
			\begin{bmatrix} 
			1 \\
			1 \\ 
			1 \\
			1 \\
			1 \\
			1\\
			1\\ 
			1\\ 
			1\\ 
			1\\ 
			1\\ 
			1\\ 
			1\\ 
			1 
			\end{bmatrix}
			;
			d = 0,85
			; 
			\text{convergence à 2 itérations}.
		\]
	\end{block}
\end{frame}		

\begin{frame}
    \frametitle{\textsc{Extraction du descripteur PageRank personnalisé}}
    \begin{block}{Variables et Paramètres}
        \tiny
        \[
        H = 
        \left[
        \begin{array}{cccccccccccccc}
            0 & 0 & 0 & 1 & 0 & 0 & 0 & 0 & 0 & 0 & 0 & 0 & 0 & 0  \\
            0 & 0 & 0 & 0 & 1 & 0 & 0 & 0 & 0 & 0 & 0 & 0 & 0 & 0 \\ 
            0 & 0 & 0 & 0 & 0 & \frac{1}{2} & \frac{1}{2} & 0 & 0 & 0 & 0 & 0 & 0 & 0  \\
            \frac{1}{2} & 0 & 0 & 0 & 0 & 0 & 0 & 0 & 0 & 0 & \frac{1}{2} & 0 & 0 & 0 \\
            0 & \frac{1}{2} & 0 & 0 & 0 & 0 & 0 & 0 & 0 & 0 & 0 & \frac{1}{2} & 0 & 0 \\
            0 & 0 & \frac{1}{2} & 0 & 0 & 0 & 0 & 0 & 0 & 0 & 0 & 0 & \frac{1}{2} & 0 \\
            0 & 0 & \frac{1}{2} & 0 & 0 & 0 & 0 & 0 & 0 & 0 & 0 & 0 & 0 & \frac{1}{2} \\ 
            0 & 0 & 0 & 0 & 0 & 0 & 0 & 0 & 0 & 0 & 0 & 0 & 1 & 0 \\ 
            0 & 0 & 0 & 0 & 0 & 0 & 0 & 0 & 0 & 0 & 0 & 1 & 0 & 0 \\ 
            0 & 0 & 0 & 0 & 0 & 0 & 0 & 0 & 0 & 0 & \frac{1}{2} & 0 & 0 & \frac{1}{2} \\  
            0 & 0 & 0 & \frac{1}{2} & 0 & 0 & 0 & 0 & 0 & \frac{1}{2} & 0 & 0 & 0 & 0 \\ 
            0 & 0 & 0 & 0 & \frac{1}{2} & 0 & 0 & 0 & \frac{1}{2} & 0 & 0 & 0 & 0 & 0 \\ 
            0 & 0 & 0 & 0 & 0 & \frac{1}{2} & 0 & \frac{1}{2} & 0 & 0 & 0 & 0 & 0 & 0 \\ 
            0 & 0 & 0 & 0 & 0 & 0 & \frac{1}{2} & 0 & 0 & \frac{1}{2} & 0 & 0 & 0 & 0
        \end{array}
        \right]
        ; \text{La matrice de transition}
        \]
    \end{block}
\end{frame}

\begin{frame}
    \frametitle{\textsc{Extraction du descripteur PageRank personnalisé}}
    
    \begin{block}{Itération 1 U1}
        \tiny
        \[
        PR = 
        \frac{{3}}{280} 
        +  
        \begin{bmatrix} 
			\frac{17}{280} \\
			\frac{17}{280} \\ 
			\frac{17}{280} \\
			\frac{17}{280} \\
			\frac{17}{280} \\
			\frac{17}{280}\\
			\frac{17}{280}\\ 
			\frac{17}{280}\\ 
			\frac{17}{280}\\ 
			\frac{17}{280}\\ 
			\frac{17}{280}\\ 
			\frac{17}{280}\\ 
			\frac{17}{280}\\ 
			\frac{17}{280}
			\end{bmatrix}
	\times
        \left[
        \begin{array}{cccccccccccccc}
            0 & 0 & 0 & 1 & 0 & 0 & 0 & 0 & 0 & 0 & 0 & 0 & 0 & 0  \\
            0 & 0 & 0 & 0 & 1 & 0 & 0 & 0 & 0 & 0 & 0 & 0 & 0 & 0 \\ 
            0 & 0 & 0 & 0 & 0 & \frac{1}{2} & \frac{1}{2} & 0 & 0 & 0 & 0 & 0 & 0 & 0  \\
            \frac{1}{2} & 0 & 0 & 0 & 0 & 0 & 0 & 0 & 0 & 0 & \frac{1}{2} & 0 & 0 & 0 \\
            0 & \frac{1}{2} & 0 & 0 & 0 & 0 & 0 & 0 & 0 & 0 & 0 & \frac{1}{2} & 0 & 0 \\
            0 & 0 & \frac{1}{2} & 0 & 0 & 0 & 0 & 0 & 0 & 0 & 0 & 0 & \frac{1}{2} & 0 \\
            0 & 0 & \frac{1}{2} & 0 & 0 & 0 & 0 & 0 & 0 & 0 & 0 & 0 & 0 & \frac{1}{2} \\ 
            0 & 0 & 0 & 0 & 0 & 0 & 0 & 0 & 0 & 0 & 0 & 0 & 1 & 0 \\ 
            0 & 0 & 0 & 0 & 0 & 0 & 0 & 0 & 0 & 0 & 0 & 1 & 0 & 0 \\ 
            0 & 0 & 0 & 0 & 0 & 0 & 0 & 0 & 0 & 0 & \frac{1}{2} & 0 & 0 & \frac{1}{2} \\  
            0 & 0 & 0 & \frac{1}{2} & 0 & 0 & 0 & 0 & 0 & \frac{1}{2} & 0 & 0 & 0 & 0 \\ 
            0 & 0 & 0 & 0 & \frac{1}{2} & 0 & 0 & 0 & \frac{1}{2} & 0 & 0 & 0 & 0 & 0 \\ 
            0 & 0 & 0 & 0 & 0 & \frac{1}{2} & 0 & \frac{1}{2} & 0 & 0 & 0 & 0 & 0 & 0 \\ 
            0 & 0 & 0 & 0 & 0 & 0 & \frac{1}{2} & 0 & 0 & \frac{1}{2} & 0 & 0 & 0 & 0
        \end{array}
        \right]
        \times
        \begin{bmatrix} 
			1 \\
			0 \\ 
			0 \\
			1\\
			0\\ 
			0\\ 
			0\\ 
			0\\ 
			0\\
			1\\
			1\\
			0\\ 
			0\\ 
			0 
			\end{bmatrix}
	=
	\begin{bmatrix} 
			\frac{1}{25} \\
			\frac{1}{100} \\ 
			\frac{1}{100} \\
			\frac{1}{10}\\
			\frac{1}{100}\\ 
			\frac{1}{100}\\ 
			\frac{1}{100}\\ 
			\frac{1}{100}\\ 
			\frac{1}{100}\\
			\frac{7}{100}\\
			\frac{7}{100}\\
			\frac{1}{100}\\ 
			\frac{1}{100}\\ 
			\frac{1}{100} 
			\end{bmatrix}
        \]
    \end{block}
\end{frame}

\begin{frame}
    \frametitle{\textsc{Extraction du descripteur PageRank personnalisé}}
    
    \begin{block}{Itération 2 -- U1}
        \tiny
        \[
        PR = 
        \frac{{3}}{280} 
        +  
        \begin{bmatrix} 
			\frac{17}{500} \\
			\frac{17}{2000} \\ 
			\frac{17}{2000} \\
			\frac{17}{200} \\
			\frac{17}{2000} \\
			\frac{17}{2000}\\
			\frac{17}{2000}\\ 
			\frac{17}{2000}\\ 
			\frac{17}{2000}\\ 
			\frac{119}{2000}\\ 
			\frac{119}{2000}\\ 
			\frac{17}{2000}\\ 
			\frac{17}{2000}\\ 
			\frac{17}{2000}
			\end{bmatrix}
	\times
        \left[
        \begin{array}{cccccccccccccc}
            0 & 0 & 0 & 1 & 0 & 0 & 0 & 0 & 0 & 0 & 0 & 0 & 0 & 0  \\
            0 & 0 & 0 & 0 & 1 & 0 & 0 & 0 & 0 & 0 & 0 & 0 & 0 & 0 \\ 
            0 & 0 & 0 & 0 & 0 & \frac{1}{2} & \frac{1}{2} & 0 & 0 & 0 & 0 & 0 & 0 & 0  \\
            \frac{1}{2} & 0 & 0 & 0 & 0 & 0 & 0 & 0 & 0 & 0 & \frac{1}{2} & 0 & 0 & 0 \\
            0 & \frac{1}{2} & 0 & 0 & 0 & 0 & 0 & 0 & 0 & 0 & 0 & \frac{1}{2} & 0 & 0 \\
            0 & 0 & \frac{1}{2} & 0 & 0 & 0 & 0 & 0 & 0 & 0 & 0 & 0 & \frac{1}{2} & 0 \\
            0 & 0 & \frac{1}{2} & 0 & 0 & 0 & 0 & 0 & 0 & 0 & 0 & 0 & 0 & \frac{1}{2} \\ 
            0 & 0 & 0 & 0 & 0 & 0 & 0 & 0 & 0 & 0 & 0 & 0 & 1 & 0 \\ 
            0 & 0 & 0 & 0 & 0 & 0 & 0 & 0 & 0 & 0 & 0 & 1 & 0 & 0 \\ 
            0 & 0 & 0 & 0 & 0 & 0 & 0 & 0 & 0 & 0 & \frac{1}{2} & 0 & 0 & \frac{1}{2} \\  
            0 & 0 & 0 & \frac{1}{2} & 0 & 0 & 0 & 0 & 0 & \frac{1}{2} & 0 & 0 & 0 & 0 \\ 
            0 & 0 & 0 & 0 & \frac{1}{2} & 0 & 0 & 0 & \frac{1}{2} & 0 & 0 & 0 & 0 & 0 \\ 
            0 & 0 & 0 & 0 & 0 & \frac{1}{2} & 0 & \frac{1}{2} & 0 & 0 & 0 & 0 & 0 & 0 \\ 
            0 & 0 & 0 & 0 & 0 & 0 & \frac{1}{2} & 0 & 0 & \frac{1}{2} & 0 & 0 & 0 & 0
        \end{array}
        \right]
        \times
        \begin{bmatrix} 
			1 \\
			0 \\ 
			0 \\
			1\\
			0\\ 
			0\\ 
			0\\ 
			0\\ 
			0\\
			1\\
			1\\
			0\\ 
			0\\ 
			0 
			\end{bmatrix}
	=
	\begin{bmatrix} 
			\frac{21}{400} \\
			\frac{1}{100} \\ 
			\frac{1}{100} \\
			\textcolor{blue}{\frac{207}{2000}}\\
			\frac{1}{100}\\ 
			\frac{1}{100}\\ 
			\frac{1}{100}\\ 
			\frac{1}{100}\\ 
			\frac{1}{100}\\
			\frac{11}{250}\\
			\textcolor{blue}{\frac{329}{4000}}\\
			\frac{1}{100}\\ 
			\frac{1}{100}\\ 
			\frac{1}{100} 
			\end{bmatrix}
        \]
    \end{block}
\end{frame}

\subsubsection{Augmentation de données}
\begin{frame}
	\frametitle{\textsc{Augmentation de données}}
	\begin{block}{Données Augmenté de descripteurs extraits de graphes}
		  % Requires the booktabs if the memoir class is not being used
		  \begin{table}[htbp]
		     \centering
		     \tiny
		     \setlength{\heavyrulewidth}{0.2pt} % Adjust the line thickness of \toprule
		     \resizebox{\textwidth}{!}{%
		     \begin{tabular}{@{} lcccccccccccr @{}} % Column formatting, @{} suppresses leading/trailing space
		        \toprule
		        ID & RENT & OWN & MORTGAGE & PERSONAL & EDUCATION & MEDICAL & VENTURE &  HOMEIMPROVEMENT & Degree & PPR & loan\_status\\
		        \midrule
		        0 & 1 & 0 & 0 & 1 & 0 & 0 & 0 & 0 & 1 & $\frac{207}{2000}$ & 1 \\ 
		        1 & 0 & 1 & 0 & 0 & 1 & 0 & 0 & 0 & 0 && 0 \\ 
		        2 & 0 & 0 & 1 & 0 & 0 & 1 & 0 & 0 & 0 && 1 \\
		        3 & 1 & 0 & 0 & 0 & 0 & 1 & 0 & 0 & 1 && 1 \\ 
		        4 & 0 & 1 & 0 & 0 & 0 & 0 & 1 & 0 & 0 && 1 \\ 
		        5 & 1 & 0 & 1 & 0 & 1 & 0 & 0 & 0 & 0 && 0 \\ 
		        6 & 1 & 0 & 0 & 0 & 0 & 1 & 0 & 0 & 1 && 0 \\ 
		        7 & 1 & 0 & 0 & 1 & 0 & 0 & 0 & 0 & 1 && 1 \\ 
		        8 & 0 & 1 & 0 & 0 & 0 & 0 & 0 & 1 & 0 && 1 \\  
		        \bottomrule
		     \end{tabular}}
		     \caption{Données augmentées}
		     \label{tab:augD}
		  \end{table}
	\end{block}
	
\end{frame}

\begin{frame}
	\frametitle{\textsc{Conclusion}}
	\Background
	\begin{center}
		{\Huge \textbf{En résumé}}
	\end{center}
\end{frame}

\begin{frame}[plain]
	\Background
	\begin{center}
		{\Huge \textbf{Merci de votre attention}}\\
		\vspace{9pt}
		Victor DJIEMBOU $^{1}$ \\
		\vspace{9pt}
		$^{1}$ \textit{Université de Yaoundé I,\\ Faculté des Sciences,\\ Département d'Informatique,\\ Étudiant Master SD}
	\end{center}
	
\end{frame}
\end{document}