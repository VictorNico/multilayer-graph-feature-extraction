\documentclass{beamer}
\usetheme{Madrid}
\usepackage[T1]{fontenc}
\usepackage[utf8]{inputenc}
\usepackage[english]{babel}
% \usepackage{beamerthemesplit} // Activate for custom appearance
\usepackage{tikz}
\usepackage{xcolor}

% \usepackage{beamerthemesplit} // Activate for custom appearance

\title[Credit Scoring]{Prédiction du risque de crédit bancaire sensible aux coûts financiers en intégrant des descripteurs extraits des graphes}
\subtitle{Préparation du CC d'INF5019}
\author{DTVNI}
\date{\today}

\begin{document}

\frame{\titlepage}

\section[Outline]{}
\frame{\tableofcontents}

\section{Titre du mémoire}
\frame
{
  \frametitle{Titre du mémoire}

  \begin{block}{Problème}    
  Prédiction du risque de crédit bancaire sensible aux coûts financiers 
  \end{block}
   \begin{block}{Méthodologie}     
   Extraction des descripteurs ou attributs de graphes
  \end{block}
   \begin{block}{Résultat}     
   pour réduire les coûts financiers des institutions préteuses
  \end{block}
   \begin{block}{[Reformulation]}     
   Prédiction du risque de crédit bancaire sensible aux coûts financiers en intégrant des descripteurs extraits des graphes pour réduire le coûts financiers des institutions prêteuses.
  \end{block}
}

\section{Questions principale et questions secondaires}
\frame
{
  \frametitle{Questions principale et questions secondaires}

  \begin{block}{Question Principale}    
  Comment proposer une approche pour améliorer la personnalisation du PageRank, prendre en compte la décision de prêt dans la modélisation de graphe multicouches, sélectionner les attributs descriptives optimales pour la construction du graphe et évaluer les coûts financiers des modèles construits ?
  \end{block}
   \begin{block}{Questions secondaires}     
   {\small
   \begin{itemize}
   	\item Comment évaluer les coûts financiers ?
	\item Comment intégrer la décision de prêt dans la modélisation sous forme de graphe multicouches ?
	\item Comment personnaliser davantage l'algorithme PageRank pour mieux refléter les spécificités du domaine d'application ?
	\item Quelles sont les attributs descriptives les plus pertinentes à inclure dans la construction du graphe ?
\end{itemize}}

  \end{block}
}

\section{Méthodologie à appliquer}
\frame
{
  \frametitle{Méthodologie à appliquer}

   \begin{block}{Étapes}     
   {\tiny
   \begin{itemize}
   	\item Définir la question principale et les questions secondaires
	\item Acquérir les données
	\item Analyser les données
	\item Prétraiter les données
	\item Construire le graphe multicouches
	\item Extraire des attributs de ce graphe
	\item Intégrer les attributs extraits dans les données originales
	\item Concevoir le modèle
	\item Entrainer le modèle
	\item Evaluer le modèle
	\item Tester le modèle
	\item Analyser les résultats
	
\end{itemize}}

  \end{block}
}

\section{Résultat Projeté}
\frame
{
  \frametitle{Résultat Projeté}

   \begin{block}{}     
   Une évaluation précoce de la solvabilité des prêts garantissant une réduction des coûts financiers par les institutions prêteuses.

  \end{block}
}

\section{Epreuve EXAMEN INF5019 - 2023}
\frame
{
  \frametitle{Epreuve EXAMEN INF5019 - 2023}

   \begin{block}{Mon sujet de mémoire de recherche}     

  \end{block}
  
  \begin{block}{Décomposition en ses différentes composantes}     

  \end{block}
  
  \begin{block}{Le résumé de mon travail en 150 mots}     

  \end{block}
  
  \begin{block}{Trois mots clés de mon sujet de recherche}     

  \end{block}
  
  \begin{block}{Mon domaine de recherche}     

  \end{block}
  
  \begin{block}{Deux articles fare de mon domaine spécifique de recherche avec justification à l'appui}     

  \end{block}
  
  \begin{block}{Deux rencontres internationaux de mon domaine de recherche}     

  \end{block}
  
  \begin{block}{Les étapes pour réaliser un bon état de l'art}     

  \end{block}
  
  \begin{block}{Deux grande qualités de chercheur}     

  \end{block}
  
  \begin{block}{Trois styles de références bibliographiques}     

  \end{block}
  
  \begin{block}{Descriptions de l'approche méthodologique}     

  \end{block}
}
\end{document}
